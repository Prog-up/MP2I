\documentclass[a4paper, 12pt, twoside]{article}


%------------------------------------------------------------------------
%
% Author                :   Lasercata
% Last modification     :   2022.03.15
%
%------------------------------------------------------------------------


%------ini
\usepackage[utf8]{inputenc}
\usepackage[T1]{fontenc}
\usepackage[french]{babel}
%\usepackage[english]{babel}


%------geometry
\usepackage[textheight=700pt, textwidth=500pt]{geometry}


%------color
\usepackage{xcolor}
\definecolor{ff4500}{HTML}{ff4500}
\definecolor{00f}{HTML}{0000ff}
\definecolor{0ff}{HTML}{00ffff}
\definecolor{656565}{HTML}{656565}

\renewcommand{\emph}{\textcolor{ff4500}}
\renewcommand{\em}{\color{ff4500}}

\newcommand{\strong}[1]{\textcolor{ff4500}{\bf #1}}
\newcommand{\st}{\color{ff4500}\bf}


%------Code highlighting
\usepackage{listings}

\definecolor{cbg}{HTML}{272822}
\definecolor{cfg}{HTML}{ececec}
\definecolor{ccomment}{HTML}{686c58}
\definecolor{ckw}{HTML}{f92672}
\definecolor{cstring}{HTML}{e6db72}
\definecolor{cstringlight}{HTML}{98980f}
\definecolor{lightwhite}{HTML}{fafafa}

\lstdefinestyle{DarkCodeStyle}{
    backgroundcolor=\color{cbg},
    commentstyle=\itshape\color{ccomment},
    keywordstyle=\color{ckw},
    numberstyle=\tiny\color{cbg},
    stringstyle=\color{cstring},
    basicstyle=\ttfamily\footnotesize\color{cfg},
    breakatwhitespace=false,
    breaklines=true,
    captionpos=b,
    keepspaces=true,
    numbers=left,
    numbersep=5pt,
    showspaces=false,
    showstringspaces=false,
    showtabs=false,
    tabsize=4,
    xleftmargin=\leftskip
}

\lstdefinestyle{LightCodeStyle}{
    backgroundcolor=\color{lightwhite},
    commentstyle=\itshape\color{ccomment},
    keywordstyle=\color{ckw},
    numberstyle=\tiny\color{cbg},
    stringstyle=\color{cstringlight},
    basicstyle=\ttfamily\footnotesize\color{cbg},
    breakatwhitespace=false,
    breaklines=true,
    captionpos=b,
    keepspaces=true,
    numbers=left,
    numbersep=10pt,
    showspaces=false,
    showstringspaces=false,
    showtabs=false,
    tabsize=4,
    frame=L,
    xleftmargin=\leftskip
}

%\lstset{style=DarkCodeStyle}
\lstset{style=LightCodeStyle}

%Usage : \begin{lstlisting}[language=Caml] ... \end{lstlisting}


%-------make the table of content clickable
\usepackage{hyperref}
\hypersetup{
    colorlinks,
    citecolor=black,
    filecolor=black,
    linkcolor=black,
    urlcolor=black
}


%------pictures
\usepackage{graphicx}
%\usepackage{wrapfig}


%------tabular
%\usepackage{color}
%\usepackage{colortbl}
%\usepackage{multirow}


%------Physics
%---Packages
%\usepackage[version=4]{mhchem} %$\ce{NO4^2-}$

%---Commands
\newcommand{\link}[2]{\mathrm{#1} \! - \! \mathrm{#2}}
\newcommand{\pt}[1]{\cdot 10^{#1}} % Power of ten
\newcommand{\dt}[2][t]{\dfrac{d#2}{d#1}} % Derivative


%------math
%---Packages
%\usepackage{textcomp}
%\usepackage{amsmath}
\usepackage{amssymb}
\usepackage{mathtools} % For abs
\usepackage{stmaryrd} %for \llbracket and \rrbracket
\usepackage{mathrsfs} %for \mathscr{x} (different from \mathcal{x})

%---Commands
%-Sets
\newcommand{\N}{\mathbb{N}} %set N
\newcommand{\Z}{\mathbb{Z}} %set Z
\newcommand{\Q}{\mathbb{Q}} %set Q
\newcommand{\R}{\mathbb{R}} %set R
\newcommand{\C}{\mathbb{C}} %set C
\newcommand{\U}{\mathbb{U}} %set U
\newcommand{\seg}[2]{\left[ #1\ ;\ #2 \right]}
\newcommand{\nset}[2]{\left\llbracket #1\ ;\ #2 \right\rrbracket}

%-Exponantial / complexs
\newcommand{\e}{\mathrm{e}}
\newcommand{\cj}[1]{\overline{#1}} %overline for the conjugate.

%-Vectors
\newcommand{\vect}{\overrightarrow}
\newcommand{\veco}[3]{\displaystyle \vect{#1}\binom{#2}{#3}} %vector + coord

%-Limits
\newcommand{\lm}[2][{}]{\lim\limits_{\substack{#2 \\ #1}}} %$\lm{x \to a} f$ or $\lm[x < a]{x \to a} f$
\newcommand{\Lm}[3][{}]{\lm[#1]{#2} \left( #3 \right)} %$\Lm{x \to a}{f}$ or $\Lm[x < a]{x \to a}{f}$
\newcommand{\tendsto}[1]{\xrightarrow[#1]{}}

%-Integral
\newcommand{\dint}[4][x]{\displaystyle \int_{#2}^{#3} #4 \mathrm{d} #1} %$\dint{a}{b}{f(x)}$ or $\dint[t]{a}{b}{f(t)}$

%-left right
\newcommand{\lr}[1]{\left( #1 \right)}
\newcommand{\lrb}[1]{\left[ #1 \right]}
\newcommand{\set}[1]{\left\{ #1 \right\}}
\newcommand{\abs}[1]{\left\lvert #1 \right\rvert} % abs{x} -> |x|
\newcommand{\floor}[1]{\left\lfloor #1 \right\rfloor}
\newcommand{\ceil}[1]{\left\lceil #1 \right\rceil}
\newcommand{\lrangle}[1]{\left\langle #1 \right\rangle}

%-Others
\newcommand{\para}{\ /\!/\ } %//
\newcommand{\ssi}{\ \Leftrightarrow \ }
\newcommand{\eqsys}[2]{\begin{cases} #1 \\ #2 \end{cases}}

\newcommand{\med}[2]{\mathrm{med} \left[ #1\ ;\ #2 \right]}  %$\med{A}{B} -> med[A ; B]$
\newcommand{\Circ}[2]{\mathscr{C}_{#1, #2}}

\renewcommand{\le}{\leqslant}
\renewcommand{\ge}{\geqslant}


%------commands
%---to quote french text
\newcommand{\simplecit}[1]{\guillemotleft$\;$#1$\;$\guillemotright}
\newcommand{\cit}[1]{\simplecit{\textcolor{656565}{#1}}}
\newcommand{\quo}[1]{\cit{\it #1}}

%---to indent
\newcommand{\ind}[1][20pt]{\advance\leftskip + #1}
\newcommand{\deind}[1][20pt]{\advance\leftskip - #1}

%---to indent a text
\newcommand{\indented}[2][20pt]{\par \ind[#1] #2 \par \deind[#1]}
\newenvironment{indt}[2][20pt]{#2 \par \ind[#1]}{\par \deind} %Titled indented env

%---title
\newcommand{\thetitle}[2]{\begin{center}\textbf{{\LARGE \underline{\emph{#1} :}} {\Large #2}}\end{center}}

%---parts
%-I
\newcommand{\mainpart}[2][$\!\!$]{\underline{\large \textbf{\emph{\textit{#1} #2}}}}
\newcommand{\bmainpart}[2][$\!\!$]{\underline{\large \textbf{\textit{#1} #2}}}
%-A
\newcommand{\subpart}[2][$\!\!$]{\underline{\bf \textit{#1} #2}}
%-1
\newcommand{\subsubpart}[2][$\!\!$]{\underline{\textsl{#1} #2}}
%-a
\newcommand{\subsubsubpart}[2][$\!\!$]{\underline{\it #1 #2}}

\renewcommand\thesection{\Roman{section}}
\renewcommand\thesubsection{\Alph{subsection}}
\renewcommand\thesubsubsection{\arabic{subsection}}


%------page style
\usepackage{fancyhdr}
\usepackage{lastpage}

\setlength{\headheight}{18pt}
\setlength{\footskip}{50pt}

\pagestyle{fancy}
\fancyhf{}
\fancyhead[LE, RO]{\textit{\today}}
\fancyhead[RE, LO]{\large{\textsl{\emph{\texttt{\jobname}}}}}

\fancyfoot[RO, LE]{\textit{\texttt{Page \thepage /\pageref{LastPage}}}}
\fancyfoot[LO, RE]{\includegraphics[scale=0.12]{/home/lasercata/Pictures/1.images_profil/logo/mieux/lasercata_logo_fly_fond_blanc.png}}


%------init lengths
\setlength{\parindent}{0pt} %To avoid using \noindent everywhere.
\setlength{\parskip}{6pt}


%---------------------------------Begin Document
\begin{document}
    
    \thetitle{Chapitre 2}{Andersen}
    
    \tableofcontents
    \newpage
    
    
    \begin{indt}{\section*{Introduction}}
        
        \addcontentsline{toc}{section}{Introduction}
        
        On conçoit Andersen seulement comme un auteur de contes pour enfants.
        
        L'ambiance de ses contes est souvent étrange, un peu triste.
        
        Andersen a écrit 156 récits, auquels on peut ajouter 17 autres textes de même nature.
        
        La partie de l'\oe uvre d'Andersen connue par le grand public est seulement une petite part de son \oe uvre.
        
        Le nom des receuils de contes ont changé au cours du temps : \textit{Contes racontés aux enfants}, \textit{Contes}, puis \textit{Histoires}, et maintenant \textit{Contes et Histoires}.
        
        Ce sont des récits courts, avec une part de fantaisie, mais on ne retrouve pas les histoires habituellement racontés dans les contes (par les autres grands auteurs de contes), ce sont de pures créations.
        
        Andersen n'est pas seulement un auteur de contes.
        
        \vspace{12pt}
        
        C'est un auteur du XIX\textsuperscript{ème} siècle, né à Odense (Danemark), issu d'une famille modeste (père cordonnier, mère lavandière), dans un milieu populaire.
        
        Son père s'enrôle en 1812 dans l'armée danoise qui combat aux côtés de Napoléon. Le Danemark fait cela dans le but d'exsiter face aux royaumes de Suède et surtout de Prusse.
        
        Son père meurt en 1816 (?).
        
        Deux ans après, sa mère se remarie. Andersen ne l'accepte pas, et part pour Copenhage, car il veut faire du théâtre, qui est l'une de ses passions (Il composera une 50aine de pièces).
        
        Il échoue dans son but car il n'a pas eu de formation, mais il est pris sous la protection de Jonas Callin, directeur du théâtre qui a vu de la capacit en lui. Ce dernier lui obtient une bourse pour faire des études supérieures.
        
        En 1822 il entre au collège, il fréquente des établissements prestigieux tels que le collège d'Elseneur, et y obtient le Bac à 23 ans.
        
        Dès la fin de ses études, il publie sa première \oe uvre littéraire, un récit de voyage : \textit{Voyage à pied du canal de Halnem à la pointe de Amager}. Il en profite pour y mettre des remarques, un peu de fantaisie, montrant ainsi qu'il a lu les grands auteurs allemands de l'époque tq E.T.A.Hoffmann.
        
        Hofmann est un romantique, et Andersen est l'un de ceux qui va introduire le romantisme au Danemark. Le romantisme à l'anglaise, à l'allemande, qui a recours au rêves et la fantaisie mais qui a recours aussi à l'ironie et l'auto ironie. Il y a aussi la dimension burlesque.
        
        A partir de cela, la vie d’Andersen va se composer de beaucoup de voyages, en 1831 il voyage en Allemagne, occasion de rencontrer des écrivains, des mécènes pour y créer un réseau littéraire et noue des amitiés auprès de mécènes pour s'y mettre en avant et s'y fait connaître. Ce sont les occasions de nourrir ses \oe uvres littéraires et de ramener des \oe uvres de voyage et impression.
        
        Lors de son voyage en All, il rencontre beaucoup d'auteurs de contes, dont Tieck et von Chamisso, auteurs de contes d'art romantique.
        
        Récit de voyage : \textit{Ombre chinoise d'un voyage dans le Harz}, influencé par Heine, grand auteur Allemand (poésie \textit{La Lorelei}).
        
        Le Harz est une région de l'Allemagne, la dernière chaine de montagne, la plus au Nord, à 300 km de la mer. C'est une zone humide, sombre.
        
        \vspace{12pt}
        
        Andersen cherche à être considéré comme un grand auteur.
        
        \vspace{12pt}
        
        Il commence ensuite à s'émanciper, en 1833, il recoit une bourse de 2 ans, il fait son premier grand voyage qui le mène à travers l'Europe : France, Suisse, Allemagne, Autriche, et surtout Italie, où il rédige ses premiers contes, à Rome, sous le soleil.
        
        Andersen est un grand romancier réaliste pour le Danemark (6 romans).
        
        Premier roman : \textit{L'improvisateur} (publié en 1835)
        
        Ses deux premiers receuils de contes racontés pour enfants aussi publiés en 1835.
        
        Au début, Andersen ne croit pas au succès des contes, il les écrit par amusement et des\oe uvrement.
        
        \`A l'époque le conte est à la mode.
        
        Voyage en Suède (1837) ;
        1840-1841 : long voyage jusqu'en Turquie, récit \textit{Le bazar d'un poète} ;
        1843 : voyage vers l'Europe (Victor Hugo, Dumas, Lamrtine, ...), Espagne, Portugal.
        
        Il passe souvent par l'Allemagne, alors éclatée en moults petits \'Etats (souvent en émulation intellectuelle), dont la principauté de Saxe-Weimar (centre pays, région de petite ville, forêt) où il rencontre le duc, dans la ville de Weimar (taille de Dole), haut lieu de la culture Allemande, dont la politique est tournée vers l'art.
        
        Le duc lui propose de s'y installer (comme Dickens qui y habite la moitié de l'année, avec qui d'ailleurs Andersen entretien une correspondance).
        
        1847 : voyage triomphal en Angleterre, rencontre avec Dickens.
        
        Andersen meurt en 1875.
        
        \vspace{12pt}
        
        \begin{indt}{L'\oe uvre d'Andersen est bien plus riche que la partie connue du grand public :}
            $-$ 6 romans ;
            
            $-$ 5 récits de voyage ;
            
            $-$ Plus de $10^3$ poèmes ;
            
            $-$ $\sim 50$ \oe uvres théâtrales ;
            
            $-$ 3 récits autobiographiques (très romancés) ;
            
            $-$ 10 volumes de journaux ;
            
            $-$ Correspondances.
        \end{indt}
        
        Il y a 18 gros volumes d'\oe uvres complètes, dont 3 seulement  sont consacrés aux contes.
        
        \vspace{12pt}
        
        \textbf{Les contes d'Andersen sont-ils bien, comme on le croit, des contes réservés aux enfants ?}
        
        L'ambiguïté vient du fait que souvent, les personnages principaux sont des enfants, le ton est enfantin, et le titre indique ``pour les enfants''.
        
    \end{indt}
    
    \vspace{12pt}
    
    \begin{indt}{\section{La voix enfantine dans les contes d'Andersen}}
        
        Andersen ne s'est jamais rêvé auteur de contes.
        
        C'est un grand romancier danois, qui a orienté le Danemark vers le romantisme.
        
        Il y a un problème de classification de ses écrits à cause des titres : contes, histoires, récits, nouvelles ?
        
        Quoi qu'il en soit, ces textes embrassent tous les styles des récits court. (comme un carnet de croquis d'un peintre).
        
        \vspace{12pt}
        
        Perraud : fin XVII\textsuperscript e siècle, début XVIII\textsuperscript e siècle.
        En s'inspirant des récits populaires, traditionnels, il rédige quelques contes, destinés à un public savant, précieux, utilisant le double sens (exemple : \textit{le petit chaperon rouge}). Ses contes sont des contes littéraires.
        
        Avec Perraud, la langue est à un niveau de clarté absolue.
        
        \vspace{12pt}
        
        Galland : traduit et invente les contes des \textit{mille et une nuits}.
        
        \vspace{12pt}
        
        Les frères Grimm sont des spécialistes des langues anciennes et de l'Allemagne et s'intéressent aux contes populaires. Ils publient l'équivalent du Litré en Allemagne. Ils voyagent à travers de nombreux villages pour écouter des contes populaires et les réécrivent pour faire un travail de collation de contes populaires.
        
        Publiés sous le titre de \textit{Contes de l'enfance et du Foyer}.
        
        \vspace{12pt}
        
        Contes d'art romantique (Eichendorff, ...) : contes d'invention qui font semblant d'avoir le ton naïf des contes dans le but de renouveler notre regard sur le réel, de retrouver notre faculté de nous émerveiller devant le monde.
        
        Romantisme : la dernière génération, celle des lumières, à trop rationalisé, donc les romantiques veulent remettre de l'émerveillement dans le monde.
        
        La présence de l'enfant dans les contes d'Andersen ne sera jamais effacée des contes.
        
        Les contes d'Andersen sont des récits \textbf{racontés à hauteur d'enfant}, une stratégie pour s'addresser aux adultes.
        
        \textit{Les nouveaux habits de l'empereur} est un bon exemple de cela, par ce conte, Andersen critique la société.
        
        \vspace{12pt}
        
        \begin{indt}{\'Etude de quatre textes :}
            $-$ \textit{Les nouveaux habits de l'empereur} p83 ;
            
            $-$ \textit{Le rossignol} (grand symbole du romantisme) p115 ;
            
            $-$ \textit{La famille heureuse}
            
            $-$ \textit{L'Ombre} p195.
        \end{indt}
        
    \end{indt}

    
    
    
\end{document}
%--------------------------------------------End
