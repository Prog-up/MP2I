\documentclass[a4paper, 12pt, twoside]{article}


%------------------------------------------------------------------------
%
% Author                :   Lasercata
% Last modification     :   2022.03.31
%
%------------------------------------------------------------------------


%------ini
\usepackage[utf8]{inputenc}
\usepackage[T1]{fontenc}
\usepackage[french]{babel}
%\usepackage[english]{babel}


%------geometry
\usepackage[textheight=700pt, textwidth=500pt]{geometry}


%------color
\usepackage{xcolor}
\definecolor{ff4500}{HTML}{ff4500}
\definecolor{00f}{HTML}{0000ff}
\definecolor{0ff}{HTML}{00ffff}
\definecolor{656565}{HTML}{656565}

\renewcommand{\emph}{\textcolor{ff4500}}
\renewcommand{\em}{\color{ff4500}}

\newcommand{\strong}[1]{\textcolor{ff4500}{\bf #1}}
\newcommand{\st}{\color{ff4500}\bf}


%------Code highlighting
\usepackage{listings}

\definecolor{cbg}{HTML}{272822}
\definecolor{cfg}{HTML}{ececec}
\definecolor{ccomment}{HTML}{686c58}
\definecolor{ckw}{HTML}{f92672}
\definecolor{cstring}{HTML}{e6db72}
\definecolor{cstringlight}{HTML}{98980f}
\definecolor{lightwhite}{HTML}{fafafa}

\lstdefinestyle{DarkCodeStyle}{
    backgroundcolor=\color{cbg},
    commentstyle=\itshape\color{ccomment},
    keywordstyle=\color{ckw},
    numberstyle=\tiny\color{cbg},
    stringstyle=\color{cstring},
    basicstyle=\ttfamily\footnotesize\color{cfg},
    breakatwhitespace=false,
    breaklines=true,
    captionpos=b,
    keepspaces=true,
    numbers=left,
    numbersep=5pt,
    showspaces=false,
    showstringspaces=false,
    showtabs=false,
    tabsize=4,
    xleftmargin=\leftskip
}

\lstdefinestyle{LightCodeStyle}{
    backgroundcolor=\color{lightwhite},
    commentstyle=\itshape\color{ccomment},
    keywordstyle=\color{ckw},
    numberstyle=\tiny\color{cbg},
    stringstyle=\color{cstringlight},
    basicstyle=\ttfamily\footnotesize\color{cbg},
    breakatwhitespace=false,
    breaklines=true,
    captionpos=b,
    keepspaces=true,
    numbers=left,
    numbersep=10pt,
    showspaces=false,
    showstringspaces=false,
    showtabs=false,
    tabsize=4,
    frame=L,
    xleftmargin=\leftskip
}

%\lstset{style=DarkCodeStyle}
\lstset{style=LightCodeStyle}

%Usage : \begin{lstlisting}[language=Caml] ... \end{lstlisting}


%-------make the table of content clickable
\usepackage{hyperref}
\hypersetup{
    colorlinks,
    citecolor=black,
    filecolor=black,
    linkcolor=black,
    urlcolor=black
}


%------pictures
\usepackage{graphicx}
%\usepackage{wrapfig}


%------tabular
%\usepackage{color}
%\usepackage{colortbl}
%\usepackage{multirow}


%------Physics
%---Packages
%\usepackage[version=4]{mhchem} %$\ce{NO4^2-}$

%---Commands
\newcommand{\link}[2]{\mathrm{#1} \! - \! \mathrm{#2}}
\newcommand{\pt}[1]{\cdot 10^{#1}} % Power of ten
\newcommand{\dt}[2][t]{\dfrac{d#2}{d#1}} % Derivative


%------math
%---Packages
%\usepackage{textcomp}
%\usepackage{amsmath}
\usepackage{amssymb}
\usepackage{mathtools} % For abs
\usepackage{stmaryrd} %for \llbracket and \rrbracket
\usepackage{mathrsfs} %for \mathscr{x} (different from \mathcal{x})

%---Commands
%-Sets
\newcommand{\N}{\mathbb{N}} %set N
\newcommand{\Z}{\mathbb{Z}} %set Z
\newcommand{\Q}{\mathbb{Q}} %set Q
\newcommand{\R}{\mathbb{R}} %set R
\newcommand{\C}{\mathbb{C}} %set C
\newcommand{\U}{\mathbb{U}} %set U
\newcommand{\seg}[2]{\left[ #1\ ;\ #2 \right]}
\newcommand{\nset}[2]{\left\llbracket #1\ ;\ #2 \right\rrbracket}

%-Exponantial / complexs
\newcommand{\e}{\mathrm{e}}
\newcommand{\cj}[1]{\overline{#1}} %overline for the conjugate.

%-Vectors
\newcommand{\vect}{\overrightarrow}
\newcommand{\veco}[3]{\displaystyle \vect{#1}\binom{#2}{#3}} %vector + coord

%-Limits
\newcommand{\lm}[2][{}]{\lim\limits_{\substack{#2 \\ #1}}} %$\lm{x \to a} f$ or $\lm[x < a]{x \to a} f$
\newcommand{\Lm}[3][{}]{\lm[#1]{#2} \left( #3 \right)} %$\Lm{x \to a}{f}$ or $\Lm[x < a]{x \to a}{f}$
\newcommand{\tendsto}[1]{\xrightarrow[#1]{}}

%-Integral
\newcommand{\dint}[4][x]{\displaystyle \int_{#2}^{#3} #4 \mathrm{d} #1} %$\dint{a}{b}{f(x)}$ or $\dint[t]{a}{b}{f(t)}$

%-left right
\newcommand{\lr}[1]{\left( #1 \right)}
\newcommand{\lrb}[1]{\left[ #1 \right]}
\newcommand{\set}[1]{\left\{ #1 \right\}}
\newcommand{\abs}[1]{\left\lvert #1 \right\rvert} % abs{x} -> |x|
\newcommand{\floor}[1]{\left\lfloor #1 \right\rfloor}
\newcommand{\ceil}[1]{\left\lceil #1 \right\rceil}
\newcommand{\lrangle}[1]{\left\langle #1 \right\rangle}

%-Others
\newcommand{\para}{\ /\!/\ } %//
\newcommand{\ssi}{\ \Leftrightarrow \ }
\newcommand{\eqsys}[2]{\begin{cases} #1 \\ #2 \end{cases}}

\newcommand{\med}[2]{\mathrm{med} \left[ #1\ ;\ #2 \right]}  %$\med{A}{B} -> med[A ; B]$
\newcommand{\Circ}[2]{\mathscr{C}_{#1, #2}}

\renewcommand{\le}{\leqslant}
\renewcommand{\ge}{\geqslant}


%------commands
%---to quote french text
\newcommand{\simplecit}[1]{\guillemotleft$\;$#1$\;$\guillemotright}
\newcommand{\cit}[1]{\simplecit{\textcolor{656565}{#1}}}
\newcommand{\quo}[1]{\cit{\it #1}}

%---to indent
\newcommand{\ind}[1][20pt]{\advance\leftskip + #1}
\newcommand{\deind}[1][20pt]{\advance\leftskip - #1}

%---to indent a text
\newcommand{\indented}[2][20pt]{\par \ind[#1] #2 \par \deind[#1]}
\newenvironment{indt}[2][20pt]{#2 \par \ind[#1]}{\par \deind} %Titled indented env

%---title
\newcommand{\thetitle}[2]{\begin{center}\textbf{{\LARGE \underline{\emph{#1} :}} {\Large #2}}\end{center}}

%---parts
%-I
\newcommand{\mainpart}[2][$\!\!$]{\underline{\large \textbf{\emph{\textit{#1} #2}}}}
\newcommand{\bmainpart}[2][$\!\!$]{\underline{\large \textbf{\textit{#1} #2}}}
%-A
\newcommand{\subpart}[2][$\!\!$]{\underline{\bf \textit{#1} #2}}
%-1
\newcommand{\subsubpart}[2][$\!\!$]{\underline{\textsl{#1} #2}}
%-a
\newcommand{\subsubsubpart}[2][$\!\!$]{\underline{\it #1 #2}}

\renewcommand\thesection{\Roman{section}}
\renewcommand\thesubsection{\arabic{subsection}}
\renewcommand\thesubsubsection{\aleph{subsection}}


%------page style
\usepackage{fancyhdr}
\usepackage{lastpage}

\setlength{\headheight}{18pt}
\setlength{\footskip}{50pt}

\pagestyle{fancy}
\fancyhf{}
\fancyhead[LE, RO]{\textit{\today}}
\fancyhead[RE, LO]{\large{\textsl{\emph{\texttt{\jobname}}}}}

\fancyfoot[RO, LE]{\textit{\texttt{Page \thepage /\pageref{LastPage}}}}
\fancyfoot[LO, RE]{\includegraphics[scale=0.12]{/home/lasercata/Pictures/1.images_profil/logo/mieux/lasercata_logo_fly_fond_blanc.png}}


%------init lengths
\setlength{\parindent}{0pt} %To avoid using \noindent everywhere.
\setlength{\parskip}{6pt}


%---------------------------------Begin Document
\begin{document}
    
    \thetitle{Chapitre 2}{Andersen}
    
    \tableofcontents
    \newpage
    
    
    \begin{indt}{\section*{Introduction}}
        
        \addcontentsline{toc}{section}{Introduction}
        
        On conçoit Andersen seulement comme un auteur de contes pour enfants.
        
        L'ambiance de ses contes est souvent étrange, un peu triste.
        
        Andersen a écrit 156 récits, auquels on peut ajouter 17 autres textes de même nature.
        
        La partie de l'\oe uvre d'Andersen connue par le grand public est seulement une petite part de son \oe uvre.
        
        Le nom des receuils de contes ont changé au cours du temps : \textit{Contes racontés aux enfants}, \textit{Contes}, puis \textit{Histoires}, et maintenant \textit{Contes et Histoires}.
        
        Ce sont des récits courts, avec une part de fantaisie, mais on ne retrouve pas les histoires habituellement racontés dans les contes (par les autres grands auteurs de contes), ce sont de pures créations.
        
        Andersen n'est pas seulement un auteur de contes.
        
        \vspace{12pt}
        
        C'est un auteur du XIX\textsuperscript{ème} siècle, né à Odense (Danemark), issu d'une famille modeste (père cordonnier, mère lavandière), dans un milieu populaire.
        
        Son père s'enrôle en 1812 dans l'armée danoise qui combat aux côtés de Napoléon. Le Danemark fait cela dans le but d'exsiter face aux royaumes de Suède et surtout de Prusse.
        
        Son père meurt en 1816 (?).
        
        Deux ans après, sa mère se remarie. Andersen ne l'accepte pas, et part pour Copenhage, car il veut faire du théâtre, qui est l'une de ses passions (Il composera une 50aine de pièces).
        
        Il échoue dans son but car il n'a pas eu de formation, mais il est pris sous la protection de Jonas Callin, directeur du théâtre qui a vu de la capacit en lui. Ce dernier lui obtient une bourse pour faire des études supérieures.
        
        En 1822 il entre au collège, il fréquente des établissements prestigieux tels que le collège d'Elseneur, et y obtient le Bac à 23 ans.
        
        Dès la fin de ses études, il publie sa première \oe uvre littéraire, un récit de voyage : \textit{Voyage à pied du canal de Halnem à la pointe de Amager}. Il en profite pour y mettre des remarques, un peu de fantaisie, montrant ainsi qu'il a lu les grands auteurs allemands de l'époque tq E.T.A.Hoffmann.
        
        Hofmann est un romantique, et Andersen est l'un de ceux qui va introduire le romantisme au Danemark. Le romantisme à l'anglaise, à l'allemande, qui a recours au rêves et la fantaisie mais qui a recours aussi à l'ironie et l'auto ironie. Il y a aussi la dimension burlesque.
        
        A partir de cela, la vie d’Andersen va se composer de beaucoup de voyages, en 1831 il voyage en Allemagne, occasion de rencontrer des écrivains, des mécènes pour y créer un réseau littéraire et noue des amitiés auprès de mécènes pour s'y mettre en avant et s'y fait connaître. Ce sont les occasions de nourrir ses \oe uvres littéraires et de ramener des \oe uvres de voyage et impression.
        
        Lors de son voyage en All, il rencontre beaucoup d'auteurs de contes, dont Tieck et von Chamisso, auteurs de contes d'art romantique.
        
        Récit de voyage : \textit{Ombre chinoise d'un voyage dans le Harz}, influencé par Heine, grand auteur Allemand (poésie \textit{La Lorelei}).
        
        Le Harz est une région de l'Allemagne, la dernière chaine de montagne, la plus au Nord, à 300 km de la mer. C'est une zone humide, sombre.
        
        \vspace{12pt}
        
        Andersen cherche à être considéré comme un grand auteur.
        
        \vspace{12pt}
        
        Il commence ensuite à s'émanciper, en 1833, il recoit une bourse de 2 ans, il fait son premier grand voyage qui le mène à travers l'Europe : France, Suisse, Allemagne, Autriche, et surtout Italie, où il rédige ses premiers contes, à Rome, sous le soleil.
        
        Andersen est un grand romancier réaliste pour le Danemark (6 romans).
        
        Premier roman : \textit{L'improvisateur} (publié en 1835)
        
        Ses deux premiers receuils de contes racontés pour enfants aussi publiés en 1835.
        
        Au début, Andersen ne croit pas au succès des contes, il les écrit par amusement et des\oe uvrement.
        
        \`A l'époque le conte est à la mode.
        
        Voyage en Suède (1837) ;
        1840-1841 : long voyage jusqu'en Turquie, récit \textit{Le bazar d'un poète} ;
        1843 : voyage vers l'Europe (Victor Hugo, Dumas, Lamrtine, ...), Espagne, Portugal.
        
        Il passe souvent par l'Allemagne, alors éclatée en moults petits \'Etats (souvent en émulation intellectuelle), dont la principauté de Saxe-Weimar (centre pays, région de petite ville, forêt) où il rencontre le duc, dans la ville de Weimar (taille de Dole), haut lieu de la culture Allemande, dont la politique est tournée vers l'art.
        
        Le duc lui propose de s'y installer (comme Dickens qui y habite la moitié de l'année, avec qui d'ailleurs Andersen entretien une correspondance).
        
        1847 : voyage triomphal en Angleterre, rencontre avec Dickens.
        
        Andersen meurt en 1875.
        
        \vspace{12pt}
        
        \begin{indt}{L'\oe uvre d'Andersen est bien plus riche que la partie connue du grand public :}
            $-$ 6 romans ;
            
            $-$ 5 récits de voyage ;
            
            $-$ Plus de $10^3$ poèmes ;
            
            $-$ $\sim 50$ \oe uvres théâtrales ;
            
            $-$ 3 récits autobiographiques (très romancés) ;
            
            $-$ 10 volumes de journaux ;
            
            $-$ Correspondances.
        \end{indt}
        
        Il y a 18 gros volumes d'\oe uvres complètes, dont 3 seulement  sont consacrés aux contes.
        
        \vspace{12pt}
        
        \textbf{Les contes d'Andersen sont-ils bien, comme on le croit, des contes réservés aux enfants ?}
        
        L'ambiguïté vient du fait que souvent, les personnages principaux sont des enfants, le ton est enfantin, et le titre indique ``pour les enfants''.
        
    \end{indt}
    
    \vspace{12pt}
    
    \begin{indt}{\section{La voix enfantine dans les contes d'Andersen}}
        
        Andersen ne s'est jamais rêvé auteur de contes.
        
        C'est un grand romancier danois, qui a orienté le Danemark vers le romantisme.
        
        Il y a un problème de classification de ses écrits à cause des titres : contes, histoires, récits, nouvelles ?
        
        Quoi qu'il en soit, ces textes embrassent tous les styles des récits court. (comme un carnet de croquis d'un peintre).
        
        \vspace{12pt}
        
        Perraud : fin XVII\textsuperscript e siècle, début XVIII\textsuperscript e siècle.
        En s'inspirant des récits populaires, traditionnels, il rédige quelques contes, destinés à un public savant, précieux, utilisant le double sens (exemple : \textit{le petit chaperon rouge}). Ses contes sont des contes littéraires.
        
        Avec Perraud, la langue est à un niveau de clarté absolue.
        
        \vspace{12pt}
        
        Galland : traduit et invente les contes des \textit{mille et une nuits}.
        
        \vspace{12pt}
        
        Les frères Grimm sont des spécialistes des langues anciennes et de l'Allemagne et s'intéressent aux contes populaires. Ils publient l'équivalent du Litré en Allemagne. Ils voyagent à travers de nombreux villages pour écouter des contes populaires et les réécrivent pour faire un travail de collation de contes populaires.
        
        Publiés sous le titre de \textit{Contes de l'enfance et du Foyer}.
        
        \vspace{12pt}
        
        Contes d'art romantique (Eichendorff, ...) : contes d'invention qui font semblant d'avoir le ton naïf des contes dans le but de renouveler notre regard sur le réel, de retrouver notre faculté de nous émerveiller devant le monde.
        
        Romantisme : la dernière génération, celle des lumières, à trop rationalisé, donc les romantiques veulent remettre de l'émerveillement dans le monde.
        
        La présence de l'enfant dans les contes d'Andersen ne sera jamais effacée des contes.
        
        Les contes d'Andersen sont des récits \strong{racontés à hauteur d'enfant}, une stratégie pour s'addresser aux adultes.
        
        \textit{Les nouveaux habits de l'empereur} est un bon exemple de cela, par ce conte, Andersen critique la société.
        
        \vspace{12pt}
        
        \begin{indt}{\'Etude de quatre textes :}
            $-$ \textit{Les nouveaux habits de l'empereur} p83 ;
            
            $-$ \textit{Le rossignol} (grand symbole du romantisme) p115 ;
            
            $-$ \textit{La famille heureuse}
            
            $-$ \textit{L'Ombre} p195.
        \end{indt}
        
        
        \vspace{12pt}
        
        Les contes d'Andersen recouvrent une grande diversité, de nombreux schémas naratifs, du simple conte populaire jusqu'à la nouvelle moderne.
        
        Mais dans tous les contes, il y a quelque chose de particulier, un ton particulier, une manière de raconter quasiment enfantine.
        
        Ces contes se caractérisent par une abondance de \emph{propositions éliptiques}, un ton proche du language parlé, abondance d'\emph{anacolutes} (ruptures dans la construction d'une phrase (ex : \cit{le saut, elle le met sur la table})), la \emph{parataxe} (abscence de liens logiques)
        
        \emph{Codes linguistiques du language parlé}, qui s'opposent à la littérature danoise, et au reste des \oe uvres d'Andersen.
        
        %A la prose classique (dans ses autres \oe uvres), il oppose
        
        Réguilièrement dans les contes, il met en scène cette rupture entre le language attendu ds une \oe uvre littéraire, et celui des contes.
        Par ex mis en scène dans \textit{Les nouveaux habits de l'empereur}, où la voix de l'enfant ramène le spectacle politique à ce qu'il est, un spectacle.
        
        La parole des enfants est un enjeux important dans les contes d'Andersen
        
        \vspace{12pt}
        
        \begin{indt}{\subsection{``Les Nouveaux Habits de l'Empereur''}}
            Curieux empereur qui s'occupe plus de sa garde-robe que de son empire.
            
            Raconté dans le passé, mais bien souvent les récits mis dans le passé dénoncent le présent.
            
            %Deux escrocs arrivent et ajoutent 
            
            \quo{tout flatteur vit au dépend de celui qui l'écoute} (La Fontaine)
            
            Les deux escrocs vendent seulement l'idée de nouveaux habits.
            
            Cette histoire traite du pouvoir de séduction de l'apparence sur l'empereur comme sur le commun des mortels. C'est cette apparence trompeuse qui au cours de l'histoire va être démasquée.
            
            Le conte va plus loin : peu importe que l'empereur soit nu ou vetu, l'important est que chaque personne qui le voit disent qu'ils voient des habits, aucun n'a le courage de dire ce qu'il voit.
            
            L'histoire traite de la vérité et du mensonge, et de la facilité à laquelle les hommes recourent au mensonge pour éviter d'affronter la vérité.
            
            La critique d'Andersen est plus une critique morale que politique, d'un monde dominé par la superficialité, par ce qui brille, et par un esprit calculateur (l'empereur aime les habits (apparence), et remplissent une fonction politique (calcul)).
            
            \`A cela, Andersen oppose la naïveté de l'enfant, qui dit ce qu'il voit sans calcul, \quo{mais \emph{voyons}, il n'a rien sur lui}.
            
            Ce conte n'a rien à voir avec une histoire enfantine : pouvoir, corruption, stupidité politique.
            
            On pourait s'attendre à voir cela dans un récit non enfantin.
            
            %Plus proche de la satire politique que d'un
            
            L'histoire a pour thème un sujet sérieux (dévoilement des illusion qui nous cachent le monde et le déforment).
            
            Là où le conte est cruel, i.e satirique, c'est comment les hommes de pouvoir, en dépit de tout bon sens, se cramponnent aux illusions auquel leur monde est attaché.
            
            On pourait dire qu'il y a une leçon de morale : les adultes sont st pris au piège des illusions, de l'escrocquerie, de la stupidité car il leur manque la naïveté propre à l'enfance, il faudrait qu'ils retrouvent l'innocence et la naïveté des enfants.
            
            Chez Andersen, la capacité d'accéder à la vérité et la simplicité enfantine vont de pair.
            
            %C'est l'innocence de l'enfant, et l'innocence du peuple
            
            Les chemins sont multiples qui mène à l'erreur (on peut tromper, se tromper)
            
            La simplicité enfantine conduit à la vérité.
            
            On voit que ds l'espace politique que construit le conte où toute parole est potentiellement une parole trompeuse, la seule posibilité pour que la vérité soit entendue, est qu'elle soit prononcée par un enfant.%,
            %ce n'est donc qu'a condition 
            
            Le problème du language est qu'il est potentiellement trompeur, même l'écrivain n'échappe pas à cette menace, soit volontairement (se met au service de qq), soit qu'il se trompe lui-même.
            
            Le but d'un écrivain est tout de même de trouver une sorte de vérité dans la parole.
            
            La parole enfantine est le moyen de contenir ce qu'il y a de potentiellement trompeur dans le language.
            
            Sans cette innocence, le language a partie liée avec les illusions et la tromperie, car la parole est svt fondée sur l'exercice d'une dissimulation.
            
            Page 86 : \quo{il ne voulait pas dire qu'il ne voyait rien}
            
            C'est donc sa liaison à l'innocence ou à la non innocence qui détermine la vérité.
            
            C'est une révolution.
            
            Avant : La capacité à dire la vérité est une capacité liée au language : recherche d'un moyen d'expression à dire ce qui est.
            
            Traquer les ambiguïtés du language permettrait de s'approcher de la vérité.
            
            Andersen montre que cela ne suffit pas : les sentiments de celui qui le dit sont tout aussi importants.
            
            Inventer un language capable de dire la vérité est donc inventer une language littéraire qui soit proche de celui des enfants, naïfs.
            
            Retrouver l'authenticité de parole perdue par les adultes
            
            $\rightarrow$ récit à \emph{auteur d'enfants}.
            
            \vspace{12pt}
            
            c'est aussi le but des frères Grimm, qui s'efforcent de retrouver comme une parole populaire originaire.
            
            Mais ils partent du principe qu'il existe une parole originaire, et que cette parole est plus proche de la vérite
            
            Andersen voit qu'il y a pb à essayer de recréer une \oe uvre savante dans une langue construite en dehors de la science.
            
            Ex : Du Buffet : art brut : regroupe \oe uvres d'enfants, de malades mentaux.
            
            Frères Grimm ne posent pas vraiement la question de cette parole prétenduement originelle.
            
            Andersen ne fait pas comme si il écrivait comme un enfant, il écrit comme un enfant, en le montrant.
            
            Là où Andersen échappe à la critique d'avoir voulu échapper à
            
            L'enfance n'est pas simplement la qualité des enfants, mais une dimension globale et générale de l'enfance.
            
            Nous nous comportons tous comme des enfants, enfants et adultes. Ce qui change, c'est notre rapport à notre enfance.
            
            %En faisant des écrits à la manière des enfants
            
            L'autre enfant dans le conte est l'empereur, mais il ne sait pas, ne voit pas qu'il se comporte comme un enfant.
            
            Il ne s'occupe de ses soldats que pour les passer en revu, comme un enfant range ses soldats de plomb.
            
            La vie de l'empereur est décrite comme celle d'un enfant.
            
            L'empereur est immature car il se comporte comme un enfance, sans le voir.
            
            Les habits qui n'existent pas montrent bien que tous ceux qui affirment les voir ne remplissent pas bien leur fonction, la leçon qu'ils prédendent donner est bien donnée.
            
            Andersen interroge le rapport que l'on a à l'enfance.
            
            Le conte fait jaillir qqch qui pourrait revigorer nos vies, ms dès que le livre est fermé, la vie de la société reprend.
        \end{indt}
        
        \vspace{12pt}
        
        \begin{indt}{\subsection{``Le rossignol''}}
            Cette ambition de retrouver une forme de vérité de parole est celle de tous les romantiques.
            
            Dans les romans de Dumas, les peuples deviennent un personnage.
            
            Intérêt des romantiques pour le rêve, les sentiments, l'inspiration de la nature (un des motifs récurrents de la poésie romantique).
            
            Le rossignol se trouve svt dans les poésies romantiques.
            
            Rossignol : oiseau gris, ne ressort pas visuellement, mais a un chant extraordinaire.
            
            Keats, poète anglais.
            
            Le chant éternel de la nature, représenté par celui du rossignol, montre au poète que la beauté de la nature existe.
            
            Tout le romantisme est dans ce poème de Keats
            
            D'un côté le pouvoir, la fausseté, et de l'autre, opposé : la pureté, l'innocence.
            
            Un rossignol, qui a su de son chant enchanter la cour d'un empereur de Chine, est bientôt remplacé par un oiseau mécanique, de belle apparence, dont le chant est plus réguilier.
            
            l'histoire, qui parle de nous, est déplacée dans un territoire lointain, et on faint d'éloigner la satire en utilisant le language enfantin, mais c'est justement cela la satire.
            
            Intentions du conte : le maître de musique prend la parole pour défendre l'oiseau mécanique, dire en quoi il est supérieur.
            
            Platon : Les sophistes prédentent savoir tout sur tout, mais sont incapables de parler de leur savoir, tandis de Platon sait qu'il ne sait rien.
            
            p121
            
            Ce qui est un défaut devient une qualité dans la bouche du maître de musique
            
            Au propos moraux et politique du conte précédant, il rajoute 
            
            À travers les propos du maître de musique, Andersen formule la critique d'un art dont le chant ne vient pas du c\oe ur.
            
            Mais cet oiseau mécanique à des limites : au moment où l'empereur est malade et a besoin du chant de l'oiseau, il ne peut pas fonctionner.
            
            C'est une enfant innocente et pauvre qui ramène la vérité auprès de l'empereur.
            
            \vspace{12pt}
            
            Les enfants ont un regard naïf sur le monde, c'est une invitation pour les adultes à retrouver se regard sur le monde, enlever le calcul, ...
            
            Leçon : il n'y a pas de justice, de santé possible si le pouvoir n'est accompagné par ce qui est pauvre et populaire, c-à-d ce qui est non corrompu par le pourvoir, et non corrompu par la société, bref s'il n'est accompagné par ce qui est enfantin.
            
            Ce caractère naturel et pauvre ds le conte du rossignol se manifeste par le chant d'un oiseau, et pas n'importe lequel : celui du rossignol, un chant exprimant la voix pure de la nature.
            
            
            Caractère enfantin, innocence, naïveté, ..., tout cela est relié pour Andersen, liées par un ordre profond
            
            Andersen oppose cet ordre naturel aux routines languagières.
            
            Fin : mort du personnage : Andersen idéaliste.
            
            Ce n'est qu'un rêve, mais que l'espace du conte parvient cepandant à faire exister.
            
            \vspace{12pt}
            
            Au delà de ces contes, il y a des contes qui vont plus développer plus d'autres aspects, comme la satire par la voix des enfants, comme dans \textit{les nouveaux habits de l'empereur}, où l'empereur est immature par rapport à l'enfant.
        \end{indt}
        
        \vspace{12pt}
        
        \begin{indt}{\subsection{``La famille heureuse''}}
            Met en scène une famille d'escargots, caricaturant une famille bourgeoise, histoire dans laquelle il ne se passe rien.
            
            Il ne se passe rien, un rien les étourdit, il se sont enfermés dans une représentation du monde dont ils sont au centre, pour eux, le monde tourne autour d'eux.
            
            Deux considérations : s'abriter de la pluie, et marrier leur fils, avec quelqu'un de leur rang (et pas  une limace par exemple, ceux-ci n'ont pas de maison sur leur dos, ils n'ont que leur force de travail, d'ailleurs la notion de travail échappe aux escargots (fourmis)).
            
            Ce texte est un modèle sur lequel sont construits d'autres textes, comme \textit{le Sapin}, \textit{la Bougie} (svt les objets et les animeaux sont imbus d'eux-mêmes dans les contes d'Andersen).
            
            Récit qui utilise un narateur qui utilise la stratégie du ton naratif enfantin.
            
            Le conte s'affiche comme aux limites du language des enfants.
            
            Il est manifeste dans ce conte que tout en maintenant son récit ds les limites de l'enfant, Andersen ne cesse d'adresser des clin d'\oe uil aux lecteurs adultes.
            
            (\textit{e.g} h jms contredits ds leur rpz du monde, dc pensent que leurs rpz métaphysiques sont vraies)
            
            Pour que l'enfant puisse comprendre ce qu'est l'oseille, il fait des comparaisons, il donne des repères avec le coprs de l'enfant.
            
            \vspace{12pt}
            
            \emph{L'inversion} : prendre comme personnage principal un objet ou annimal, qui est habituellement au second plan, qui existent non pas pour eux même, mais pour les h (ex : on pourrait faire l'histoire de l'épée de Zoro), et prendre le point de vue de ces objets.
            
            Ds notre monde, on possède des ciseaux pour découper du papier. Dans les contes, les choses sont \textit{inversées} : on voit les choses du point de vue des objets (ex : la bougie pense que le monde a été crée pour qu'elle puisse l'éclairer).
            
            Satire : est-ce que les hommes ne font pas la même chose ?
            
            Nous organisons tout notre monde en fontion de l'expérience que nous pouvons en avoir en tant qu'espèce, ou même, en tant qu'individu.
            
            Par exemple dans le \textit{Faux col}, il y a beaucoup d'allusions érotiques, que l'enfant l'entend pas, et le faux col se prend pour le cavalier.
        \end{indt}
        
        \vspace{12pt}
        
        \begin{indt}{\subsection{``L'Ombre''}}
            Un des grands réçits fantastiques de cette génération de romantiques.
            
            \vspace{12pt}
            
            Il ne faut pas réduire les contes d'Andersen à une opposition entre l'innocence de l'enfant aux mensonges et illusions du monde des adultes.
            
            Il y a des contes très sombres, comme \textit{l'ombre}.
            
            Met en scène deux personnages, un savant et son ombre.
            
            Le savant voyage en Italie (surement un double d'Andersen, ou de certains idéaux d'Andersen), et durant ce voyage, il se détache de son ombre.
            
            Histoire commence avec un ton enfantin.
            
            Le savant est venu là pour étudier l'art classique, venu du nord, comme il y en a tant, un de ces hommes qui a ressentit l'appel de l'Italie, c-à-d de l'antiquité, de l'art classique, antique.
            
            Goethe, homme savant, homme complet, courant pré-romantique (Sturm-und-Drang), prend la fuite (h de responsabilités) en Italie, il en revient (après un an) avec d'autres idéaux, tournant classique.
            
            Dans les pays germaniques, la référence à Goethe est importante, raison pour laquelle de nombreux artistes voyagent en Italie.
            
            Goethe défend l'idéal d'un homme en harmonie avec lui-même, avec ses valeurs, et avec le monde.
            
            Rêve Goethéen : rêve d'une culture où la recherche du beau, la conduite de soi, la conduite des affaires politique et la quête de la vérité forment une unité.
            
            Idéal platonicien du vrai, du beau, du bon.
            
            %Goethe a illustré cela en organisant son pays
            
            \vspace{12pt}
            
            L'idéal Goethéen n'est qu'un rêve, rêve dont va mourrir le savant dans le conte.
            
            L'ombre réussit très bien, mais le savant, en recherche de l'idéal goethéen, n'est pas heureux, ses livres ne trouvent pas leur public.
            
            p204
            
            Andersen nous dit que le vieux rêve d'une culture unifiée, dans laquelle la morale, la connaissance, et l'art s'est craquellée.
            
            Le conte montre qu'un monde ne respectant pas ces idéaux est en train de grandir, ce nouveaux monde représenté par l'ombre.
            
            L'ombre est devenue riche par des moyens malhonnêtes, elle a tourné à son avantage ses connaissances.
            
            Bref, l'ombre est réaliste, elle sait que le monde n'est pas ce royaume céleste dont rêve le savant.
            
            Andersen renverse l'allégorie de la caverne (on prend pour réalité des apparences, des images de la véritable réalité des choses). En effet, dans le conte, c'est l'ombre qui l'emporte, car l'ombre a un savoir que le savant n'a pas : la vision du monde que le savant dévellope n'est pas la réalité du monde \quo{vous ne comprenez pas le monde} (p204).
            
            Le mal est partout, y compris chez les enfants.
            
            Dans un tel monde, seul l'ombre peut prospérer.
            
            Nous vivons dans un monde qui a renoncé à ses rêves, aux rêves d'unité.
            
            L'idéal platonicien, goethéen est incarné par les enfants dans les contes d'Andersen.
            
            On l'enferme dans la catégorie d'auteur pour enfant pour ne pas voir cela.
            
            Ce monde est un md dans lequel nous ne sommes plus que l'ombre de nous-mêmes, où l'ombre triomphe.
            
            Les nazi ont implanté des camps de concentration vers les hauts lieux de la culture.
            
            Ce conte voit l'ombre croître dans l'Europe.
            
            Il ne faudrait pas oublier que ce triomphe du mal ne vient pas d'autre part que  de nous-mêmes. L'ombre qui triomphe est l'ombre du savant lui-même.
            
            Il ne s'agit pas de mettre le savant et l'ombre sur le même plan.
            
            L'erreur du savant est de ne pas avoir vu que l'ombre était en train de triompher, de l'avoir laissé partir.
            
            
            L'Ombre a vu la part d'ombre que nous portons en nous, et c'est ce qui en a fait un homme.
            
            \vspace{12pt}
            
            Trois remarques :
            
            $-$ Dans ce conte, Andersen ne considère pas le caractère enfantin comme synonyme d'innocence. On est dans l'univers d'un conte pesimiste, et la vision pesimiste du conte est telle que les images les plus sacrées s'y trouvent disloqués, même les parents, et même les enfants se comportent de la façon la plus innimaginable.
            
            Le rêve d'une victoire de l'innocence, qui serait la clé pour dénoncer le monde des adultes, devient problématique, si les enfants, sous leur apparente innocence sont vils, alors il n'y a plus d'issue à ce md des adultes.
            
            \vspace{12pt}
            
            $-$ Ce savoir inquiétant dont l'ombre a pris connaissance, elle y accéde grâce à la Poésie. Andersen nous sugère qu'il y a un lien entre la poésie, la connaissance et la morale, est la négation même du savant dans ses rêves.
            
            Le vil savoir, la poésie et la morale marchent de concert, toute autre histoire que par ex dans le \textit{Rossignol}.
            
            Un partie de l'\oe uvre d'Andersen met en scène des rêves. Le rêve d'un monde ou la beauté, la sincérité, la morale règnent.
            Ce rêve est conscient de n'être qu'un rêve.
            
            Mais il y a dans l'\oe uvre d'Andersen une autre partie qui a compris que c'est la triste banalité, les idéaux bas, la corruption, le mal qui règne dans le monde, (conception plus moderne).
            
            Poétique de la laideur (Baudelaire, ex : \textit{Une charogne})
            
            Il y a un Andersen romantique qui est conscient de créer des rêves, qui ne sont que des rêves (ex : \textit{la petite Ida})
            
            %Et il y a un Andersen qui fait signe du côté 
            
            \vspace{12pt}
            
            $-$ Dans un conte comme \textit{L'Ombre}, Andersen met en scène un dialogue avec lui-même, car le savant est un double de l'auteur, et les rêves d'harmonie du savant sont ceux du romantique Andersen.
            
            Le clivage entre le savant et l'ombre explique pourquoi le savant ne saura jms rien de son lien de parenté avec l'ombre.
            
            Si le savant pert son ombre, c'est car il ne veut pas reconnaître sa part d'ombre, et ses mauvais côtés (refus de voir la réalité).
            
            L'h moderne est un être fondamentalement divisé en lui-même (pas forcément mauvais : Soyinka : grand lettré langue anglaise, et attaché à son terroir). \'Ecart entre les aspirations et la réalité du monde. Le problème du savant est de ne pas voir la réalité du monde. Le pb de l'ombre est d'adhérer totalement à la réalité du monde.
            
            Le savant est ignorant car il ne veut pas reconnaître le monde tel qu'il est.
            
            L'ombre est reconnue comme un h car elle connait un certain nombre de connaissances essentielles.
            
            Alors que ds \textit{les nv habits de l'empereur}, on apprenait que la naïveté peut vaincre la dissimulation (d'où ds ce conte la promotion des valeurs enfantines). Ds l'ombre au contraire, il ns dit que la dissimulation est invaincible, et que l'humanité n'est plus qu'un lointain souvenir perdu, comme ntr enfance que ns regrettons peut-être, mais qui est finie.
            
            \vspace{12pt}
            
            Ce n'est pas une évolution, Andersen tient les deux vision au fil de ses \oe uvres.
            
            A la fois nous espérons et nous désperons. Mais on n'a pas tj vécu comme cela. On a longtemps pensé qu'il y avait un sens (décadence, ou pas)
            
            Le conte d'Andersen est subtil en ce qu'il ne nous invite pas à choisir entre l'ombre et le savant. Il y a qqchose de ludique ds ce conte, les deux personnages ont leurs côtés positifs et nég.
            
            L'ignorance du savant peut être l'occasion de développer un regard critique à l'égard de la philosophie, mais aussi une figure nostalegique (c'est si beau ce rêve ...)
            
            Aussi pour l'ombre : d'un côté elle a une attitude intolérable, son cynisme, mais d'un autre côté, elle est clairevoyante.
            
            Andersen ne ns permet pas de choisir entre les deux, mais plutôt d'habiter cet entre-deux.
            
            La vision dévelopée par Andersen est un md déchirée entre la nostalegie d'un idéal perdu, et la connaissance du monde comme il va.
            
            Le savant est l'incarnation de ce rêve perdu, l'ombre celle de ce que l'on pourrait appeler le malaise dans la civilisation.
            
            \vspace{12pt}
            
            Le ton naïf du début montre bien que c'est d'un pov enfantin par lequel le conte est raconté, mais la parole enfantine n'est plus celle qui va permettre de dénoncer le md adulte, mais est une technique visant à tenir ensemble les deux vision contradictoires qui habitent la conscience contemporaine, le rêve d'innocence, de bonheur, de vérité, qu'incarne le savant d'un côté, et de l'autre que nos rêve ne sont que des illusions.
            
            On peut reconstruire une morale sur cette base, et les rêves ne sont pas innutiles, qu'en bien même se ne sont que des rêves.
        \end{indt}
    \end{indt}
    
    \vspace{12pt}
    
    \begin{indt}{\section{Être un enfant : l'expérience enfantine dans les \textit{Contes} d'Andersen}}
        
        \begin{indt}{\subsection{L'enfance sous le regard des adultes}}
            Contes d'Andersen st important ds l'histoire pr comprendre ce que c'est qu'un enfant.
            
            L'enfance intéresse les adultes, elle prende de plus en plus d'importance.
            
            L'enfant n'a pas besoin du regard des adultes pour vivre.
            
            \vspace{12pt}
            
            On a svt reproché à Andersen de donner une vision stéréotypée des enfants dans ces contes. Ces enfants seraient selement des enfants tels que les adultes les voient, ou tels qu'ils voudraient qu'il soient.
            
            Il est vrai que svt, Andersen met en scène des stéréotypes, met c'est voulu.
            
            %En faisant cela, il
            
            Stéréotype : %représenatation consistant à enfermer une réalité dans
            
            essentialiser un être à quelques caractéristiques
            
            On part d'une caractéristique contingente, une parmi de multiples, et on en déduit une essence, une nature.
            
            Il réduit une réalité à l'une de ces caractéristiques.
        \end{indt}
        
    \end{indt}
    
    
    
\end{document}
%--------------------------------------------End
