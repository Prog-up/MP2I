\documentclass[a4paper, 12pt, twoside]{article}


%------------------------------------------------------------------------
%
% Author                :   Lasercata
% Last modification     :   2022.01.20
%
%------------------------------------------------------------------------


%------ini
\usepackage[utf8]{inputenc}
\usepackage[T1]{fontenc}
%\usepackage[french]{babel}
%\usepackage[english]{babel}


%------geometry
\usepackage[textheight=700pt, textwidth=500pt]{geometry}


%------color
\usepackage{xcolor}
\definecolor{ff4500}{HTML}{ff4500}
\definecolor{00f}{HTML}{0000ff}
\definecolor{0ff}{HTML}{00ffff}
\definecolor{656565}{HTML}{656565}

\renewcommand{\emph}{\textcolor{ff4500}}
\renewcommand{\em}{\color{ff4500}}

\newcommand{\strong}[1]{\textcolor{ff4500}{\bf #1}}
\newcommand{\st}{\color{ff4500}\bf}


%------Code highlighting
\usepackage{listings}

\definecolor{cbg}{HTML}{272822}
\definecolor{cfg}{HTML}{ececec}
\definecolor{ccomment}{HTML}{686c58}
\definecolor{ckw}{HTML}{f92672}
\definecolor{cstring}{HTML}{e6db72}
\definecolor{cstringlight}{HTML}{98980f}
\definecolor{lightwhite}{HTML}{fafafa}

\lstdefinestyle{DarkCodeStyle}{
    backgroundcolor=\color{cbg},
    commentstyle=\itshape\color{ccomment},
    keywordstyle=\color{ckw},
    numberstyle=\tiny\color{cbg},
    stringstyle=\color{cstring},
    basicstyle=\ttfamily\footnotesize\color{cfg},
    breakatwhitespace=false,
    breaklines=true,
    captionpos=b,
    keepspaces=true,
    numbers=left,
    numbersep=5pt,
    showspaces=false,
    showstringspaces=false,
    showtabs=false,
    tabsize=4,
    xleftmargin=\leftskip
}

\lstdefinestyle{LightCodeStyle}{
    backgroundcolor=\color{lightwhite},
    commentstyle=\itshape\color{ccomment},
    keywordstyle=\color{ckw},
    numberstyle=\tiny\color{cbg},
    stringstyle=\color{cstringlight},
    basicstyle=\ttfamily\footnotesize\color{cbg},
    breakatwhitespace=false,
    breaklines=true,
    captionpos=b,
    keepspaces=true,
    numbers=left,
    numbersep=10pt,
    showspaces=false,
    showstringspaces=false,
    showtabs=false,
    tabsize=4,
    frame=L,
    xleftmargin=\leftskip
}

%\lstset{style=DarkCodeStyle}
\lstset{style=LightCodeStyle}

%Usage : \begin{lstlisting}[language=Caml] ... \end{lstlisting}


%-------make the table of content clickable
\usepackage{hyperref}
\hypersetup{
    colorlinks,
    citecolor=black,
    filecolor=black,
    linkcolor=black,
    urlcolor=black
}


%------pictures
\usepackage{graphicx}
%\usepackage{wrapfig}


%------tabular
%\usepackage{color}
%\usepackage{colortbl}
%\usepackage{multirow}


%------Physics
%---Packages
%\usepackage[version=4]{mhchem} %$\ce{NO4^2-}$

%---Commands
\newcommand{\link}[2]{\mathrm{#1} \! - \! \mathrm{#2}}
\newcommand{\pt}[1]{\cdot 10^{#1}} % Power of ten
\newcommand{\dt}[2][t]{\dfrac{d#2}{d#1}} % Derivative


%------math
%---Packages
%\usepackage{textcomp}
%\usepackage{amsmath}
\usepackage{amssymb}
\usepackage{mathtools} % For abs
\usepackage{stmaryrd} %for \llbracket and \rrbracket
\usepackage{mathrsfs} %for \mathscr{x} (different from \mathcal{x})

%---Commands
%-Sets
\newcommand{\N}{\mathbb{N}} %set N
\newcommand{\R}{\mathbb{R}} %set R
\newcommand{\C}{\mathbb{C}} %set C
\newcommand{\U}{\mathbb{U}} %set U
\newcommand{\set}[2]{\left[ #1\ ;\ #2 \right]}
\newcommand{\nset}[2]{\left\llbracket #1\ ;\ #2 \right\rrbracket}

%-Exponantial / complexs
\newcommand{\e}[1]{\mathrm{e}^{#1}}
\newcommand{\ex}{\e{x}}
\newcommand{\cj}[1]{\overline{#1}} %overline for the conjugate.

%-Vectors
\newcommand{\vect}{\overrightarrow}
\newcommand{\veco}[3]{\displaystyle \vect{#1}\binom{#2}{#3}} %vector + coord

%-Limits
\newcommand{\lm}[2][{}]{\lim\limits_{\substack{#2 \\ #1}}} %$\lm{x \to a} f$ or $\lm[x < a]{x \to a} f$
\newcommand{\Lm}[3][{}]{\lm[#1]{#2} \left( #3 \right)} %$\Lm{x \to a}{f}$ or $\Lm[x < a]{x \to a}{f}$
\newcommand{\tendsto}[1]{\xrightarrow[#1]{}}

%-Integral
\newcommand{\dint}[4][x]{\displaystyle \int_{#2}^{#3} #4 \mathrm{d} #1} %$\dint{a}{b}{f(x)}$ or $\dint[t]{a}{b}{f(t)}$

%-Others
\newcommand{\para}{\ /\!/\ } %//
\newcommand{\ssi}{\ \Leftrightarrow \ }
\newcommand{\abs}[1]{\left\lvert #1 \right\rvert} % abs{x} -> |x|
\newcommand{\eqsys}[2]{\begin{cases} #1 \\ #2 \end{cases}}

\newcommand{\med}[2]{\mathrm{med} \left[ #1\ ;\ #2 \right]}  %$\med{A}{B} -> med[A ; B]$
\newcommand{\Circ}[2]{\mathscr{C}_{#1, #2}}

\newcommand{\lr}[1]{\left( #1 \right)}
\newcommand{\lrb}[1]{\left[ #1 \right]}


%------commands
%---to quote french text
\newcommand{\simplecit}[1]{\guillemotleft$\;$#1$\;$\guillemotright}
\newcommand{\cit}[1]{\simplecit{\textcolor{656565}{#1}}}
\newcommand{\quo}[1]{\cit{\it #1}}

%---to indent
\newcommand{\ind}[1][20pt]{\advance\leftskip + #1}
\newcommand{\deind}[1][20pt]{\advance\leftskip - #1}

%---to indent a text
\newcommand{\indented}[2][20pt]{\par \ind[#1] #2 \par \deind[#1]}
\newenvironment{indentedenv}[1][20pt]{\par \ind[#1]}{\par \deind}
\newenvironment{indt}[2][20pt]{#2 \begin{indentedenv}[#1]}{\end{indentedenv}} %Titled indented env

%---title
\newcommand{\thetitle}[2]{\begin{center}\textbf{{\LARGE \underline{\emph{#1} :}} {\Large #2}}\end{center}}

%---parts
%-I
\newcommand{\mainpart}[2][$\!\!$]{\underline{\large \textbf{\emph{\textit{#1} #2}}}}
\newcommand{\bmainpart}[2][$\!\!$]{\underline{\large \textbf{\textit{#1} #2}}}
%-A
\newcommand{\subpart}[2][$\!\!$]{\underline{\bf \textit{#1} #2}}
%-1
\newcommand{\subsubpart}[2][$\!\!$]{\underline{\textsl{#1} #2}}
%-a
\newcommand{\subsubsubpart}[2][$\!\!$]{\underline{\it #1 #2}}

%math part
\newcommand{\secpart}[1]{.\underline{\it #1 :}}

\newenvironment{mathdef}[2][20pt]{
    \secpart{#2} \begin{indentedenv}[#1]}
    {\end{indentedenv}}


%------page style
\usepackage{fancyhdr}
\usepackage{lastpage}

\setlength{\headheight}{18pt}
\setlength{\footskip}{50pt}

\pagestyle{fancy}
\fancyhf{}
\fancyhead[LE, RO]{\textit{\today}}
\fancyhead[RE, LO]{\large{\textsl{\emph{\texttt{\jobname}}}}}

\fancyfoot[RO, LE]{\textit{\texttt{Page \thepage /\pageref{LastPage}}}}
\fancyfoot[LO, RE]{\includegraphics[scale=0.12]{/home/lasercata/Pictures/1.images_profil/logo/mieux/lasercata_logo_fly_fond_blanc.png}}


%------init lengths
\setlength{\parindent}{0pt} %no \noindent needed !!!
\setlength{\parskip}{3pt}


%---------------------------------Begin Document
\begin{document}

    \thetitle{Maths}{Algebra}
    
    \tableofcontents
    \newpage
    
    \begin{indt}{\section{Loi de composition interne :}}
        
        \begin{indt}{\subsection{Définition}}
            Loi $*$ de composition interne sur $X$ :
            \[
                \begin{array}{rcccl}
                    * &:& X^2 &\longrightarrow& X
                    \\
                    &&(x, y) &\longmapsto& x * y
                \end{array}
            \]
        \end{indt}
        
        \vspace{6pt}
        
        \begin{indt}{\subsection{Propriétés}}
            Pour une LCI $* \in X^X$ :
            
            \vspace{6pt}
            
            $\bullet$ Associativité : $\forall (x, y, z) \in X^3,\ x*(y*z) = (x*y)*z$
            
            \vspace{6pt}
            
            $\bullet$ Commutativité : $\forall (x, y) \in X^2,\ x*y = y*x$
            
            \vspace{6pt}
            
            $\bullet$ \'Elément neutre : $\exists e \in X\ |\ \forall x \in X,\ x*e = e*x = x$
            
            \vspace{6pt}
            
            $\bullet$ \'Elément régulier :
            $
                \exists a \in X\ |\ \forall (x, y) \in X^2,\
                \begin{cases}
                    a*x = a*y \Rightarrow x = y \quad \text{régulier à gauche}
                    \\
                    x*a = y*a \Rightarrow x = y \quad \text{régulier à droite}
                \end{cases}
            $
            
            \vspace{6pt}
            
            $\bullet$ Symétrie : $x \in X$ est symétrisable $\ssi \exists x' \in X\ |\ x*x' = x'*x = e$
            
            \vspace{6pt}
            
            $\bullet$ Stabilité : $Y \in \mathcal P(X)$ stable par $* \ssi \forall (x, y) \in Y^2,\ x*y \in Y$
        \end{indt}
        
    \end{indt}
    
    \vspace{6pt}
    
    \begin{indt}{\section{Groupes}}
        \begin{indt}{\subsection{Définition (groupe)}}
            \begin{indt}{Le couple $(G, *)$ est un groupe si :}
                \begin{tabular}{ll}
                    $\bullet$ $G \neq \varnothing$
                    \vspace{6pt}
                    \\
                    $\bullet$ $\forall (x, y) \in G^2,\ x * y \in G \quad$
                    & ($*$ LCI)
                    \vspace{6pt}
                    \\
                    $\bullet$ $\forall (x, y, z) \in G^3,\ x * (y * z) = (x * y) * z \quad$
                    & ($*$ associative)
                    \vspace{6pt}
                    \\
                    $\bullet$ $\exists e \in G\ |\ \forall x \in G,\ x * e = e * x = x \quad$
                    & (élément neutre)
                    \vspace{6pt}
                    \\
                    $\bullet$ $\forall x \in G,\ \exists x' \in G\ |\ x' * x = x * x' = e$
                    & (Tout élément est symétrisable)
                \end{tabular}
                
                \vspace{12pt}
                
                On note $x^{-1} = x'$.
                
                Le groupe $(G, *)$ est dit \textit{abélien} si $*$ est commutative.
            \end{indt}
        \end{indt}
        
        \vspace{6pt}
        
        \begin{indt}{\subsection{Définition (sous-groupe)}}
            \begin{indt}{$H$ est un sous-groupe de $(G, *)$ si :}
                \begin{tabular}{ll}
                    $\bullet$ $H \in \mathcal P(G) \backslash \varnothing$
                    \vspace{6pt}
                    \\
                    $\bullet$ $\forall (x, y) \in H^2,\ x * y \in H$
                    & ($H$ stable par $*$)
                    \vspace{6pt}
                    \\
                    $\bullet$ $\forall x \in H,\ x^{-1} \in H$
                    & ($H$ stable par passage au symétrique)
                \end{tabular}
            \end{indt}
        \end{indt}
    \end{indt}
    
    \vspace{6pt}
    
    \begin{indt}{\section{Anneaux}}
        
        \begin{indt}{\subsection{Définition (anneau)}}
            \begin{indt}{Le triplet $(A, \oplus, \otimes)$ est un anneau si :}
                \begin{tabular}{ll}
                    $\bullet$ $(A, \oplus)$ est un groupe abélien
                    \vspace{6pt}
                    \\
                    $\bullet$ $\forall (x, y) \in A^2,\ x \otimes y \in A$
                    & ($\otimes$ LCI sur $A$)
                    \vspace{6pt}
                    \\
                    $\bullet$ $\forall (x, y, z) \in A^3,\ x \otimes (y \otimes z) = (x \otimes y) \otimes z$
                    & ($\otimes$ associative)
                    \vspace{6pt}
                    \\
                    $\bullet$ $\exists e \in A\ |\ \forall x \in A,\ x \otimes e = e \otimes x = x$
                    & ($\otimes$ admet un élément neutre)
                    \vspace{6pt}
                    \\
                    $\bullet$
                    $
                        \forall (x, y, z) \in A^3,\
                        \begin{cases}
                            x \otimes (y \oplus z) = (x \otimes y) \oplus (x \otimes z)
                            \\
                            (y \oplus z) \otimes x = (y \otimes x) \oplus (z \otimes x)
                        \end{cases}
                    $
                    & ($\otimes$ distributive sur $\oplus$)
                \end{tabular}
            \end{indt}
            
            \vspace{12pt}
            
            L'élément neutre de $\oplus$ est noté $0$.
            
            L'élément neutre de $\otimes$ est noté $1$.
            
            Le symétrique de $x \in A$ par $\oplus$ est noté $-x$, et appelé opposé.
            
            Le symétrique de $x \in A$ par $\otimes$, s'il existe, est noté $x^{-1}$, et appelé inverse.
            
            On définit : $\forall (x, y) \in A^2,\ xy = x \otimes y$.
            
            On définit : $A^* = \left\{ x \in A\ |\ \exists y \in A\ |\ xy = 1 \right\}$
            
            \vspace{6pt}
            
            L'anneau $(A, +, \times)$ est dit \textit{abélien} si $\times$ est commutative.
        \end{indt}
        
        \vspace{6pt}
        
        \begin{indt}{\subsection{Diviseur de zéro}}
            Soit $(A, +, \times)$ un anneau.
            
            Alors $x \in A$ est un diviseur de zéro si
            $
                \begin{cases}
                    x \neq 0
                    \\
                    \exists y \in A \setminus \{0\}\ |\ xy = 0\ \text{ou}\ yx = 0
                \end{cases}
            $.
        \end{indt}
        
        \vspace{6pt}
        
        \begin{indt}{\subsection{Anneau intègre}}
            Un anneau $(A, +, \times)$ est intègre si
            $
                \begin{cases}
                    A \neq \{ 0 \}
                    \\
                    \forall (x, y) \in A^2,\ xy = yx
                    \\
                    \forall (x, y) \in A^2,\ xy = 0 \Rightarrow (x = 0\ \text{ou}\ y = 0)
                \end{cases}
            $
        \end{indt}
        
        \vspace{6pt}
        
        \begin{indt}{\subsection{Propriétés}}
            Soit $(A, +, \times)$ un anneau. Alors :
            
            \vspace{6pt}
            
            $\bullet$ $(A^*, \times)$ est un groupe
            
            $\bullet$
            $
                \forall (x, y) \in {A^*}^2,\
                \begin{cases}
                    xy \in A^*
                    \\
                    (xy)^{-1} = y^{-1} x^{-1}
                \end{cases}
            $
            
            $\bullet$
            $
                \forall (a, b) \in A^2\ |\ ab = ba,\
                \forall n \in \N,\quad
                \begin{cases}
                    \displaystyle
                    (a + b)^n = \sum_{k = 0}^n \binom{n}{k} a^k b^{n - k}
                    \\
                    \displaystyle
                    a^n - b^n = (a - b) \sum_{k = 0}^{n - 1} a^k b^{n - 1 - k}
                \end{cases}
            $
        \end{indt}
        
        \vspace{6pt}
        
        \begin{indt}{\subsection{Définition (sous-anneau)}}
            \begin{indt}{$B$ est un sous-anneau de $(A, +, \times)$ si :}
                $\bullet$ $B$ est un sous-groupe de $(A, +)$
                
                $\bullet$ $1 \in B$
                
                $\bullet$ $\forall (x, y) \in B^2,\ xy \in B$
            \end{indt}
        \end{indt}
        
    \end{indt}
    
    \vspace{12pt}
    
    \begin{indt}{\section{Corps}}
        
        \begin{indt}{\subsection{Définition (corps)}}
            \begin{indt}{Le triplet $(K, +, \times)$ est un corps si}
                \begin{tabular}{ll}
                    $\bullet$ $(K, +, \times)$ est un anneau abélien
                    \vspace{6pt}
                    \\
                    $\bullet$ $\exists x, y \in K\ |\ x \neq y$
                    & ($K$ contient au moins deux éléments)
                    \vspace{6pt}
                    \\
                    $\bullet$ $K^* = K \setminus \{0\}$
                    & (Tous les éléments sauf $0$ sont inversibles)
                \end{tabular}
            \end{indt}
        \end{indt}
        
        \vspace{6pt}
        
        \begin{indt}{\subsection{Définition (sous-corps)}}
            \begin{indt}{$C$ est un sous-corps de $(K, +, \times)$ si :}
                \begin{tabular}{ll}
                    $\bullet$ $C$ est un sous-anneau de $(K, +, \times)$
                    \vspace{6pt}
                    \\
                    $\bullet$ $\forall x \in C \setminus \{0\},\ x^{-1} \in C$
                    & ($C$ stable par passage à l'inverse)
                \end{tabular}
            \end{indt}
        \end{indt}
        
    \end{indt}
    
    \vspace{12pt}
    
    \begin{indt}{\section{Morphismes}}
        \begin{indt}{\subsection{Morphismes de groupes}}
            \begin{indt}{\subsubsection{Définition}}
                Soient $(G, *)$ et $(G', *')$ deux groupes.
                
                \vspace{6pt}
                
                $\bullet$ Un morphisme de groupes de $G$ vers $G'$ est une fonction
                
                \[ f : G \longrightarrow G'\ |\ \forall (x, y) \in G,\ f(x * y) = f(x) *' f(y) \]
                
                $\bullet$ Un isomorphisme de groupes est un morphisme de groupes bijectif.
                
                $\bullet$ Un automorphisme de groupes est un isomorphisme de groupes d'un groupe dans lui-même.
            \end{indt}
            
            \vspace{6pt}
            
            \begin{indt}{\subsubsection{Propriétés}}
                Soient $(G, *),\ (G', *')$ deux groupes, $e,\ e'$ leurs éléments neutres respectifs, et $f$ un morphisme de groupes de $G$ vers $G'$.
                
                \vspace{6pt}
                
                $\bullet$ $f(e) = e'$
                
                $\bullet$ $\forall x \in G,\ f(x^{-1}) = \lr{f(x) \vphantom{\dfrac a b}}^{-1}$
                
                $\bullet$ La composition de deux morphismes de groupes est un morphisme de groupes.
            \end{indt}
        \end{indt}
        
        \vspace{6pt}
        
        \begin{indt}{\subsection{Noyau}}
            \begin{indt}{\subsubsection{Définition}}
                Soient $(G, *)$, $(G', *')$ deux groupes, $f$ un morphisme de groupes de $G$ vers $G'$ et $e'$ l'élément neutre de $(G', *')$.
                
                Le noyau de $f$ est :
                    \[ \mathrm{Ker}(f) = f^{-1}\lr{\{e'\} \vphantom{\dfrac a a}} = \{ x \in G\ |\ f(x) = e' \} \]
            \end{indt}
            
            \vspace{6pt}
            
            \begin{indt}{\subsubsection{Propriétés}}
                Soient $(G, *)$, $(G', *')$ deux groupes, $f$ un morphisme de groupes de $G$ vers $G'$ et $e$ l'élément neutre de $(G, *)$.
                
                Alors :
                    \[ f\ \text{injective} \ssi \mathrm{Ker}(f) = \{e\} \]
            \end{indt}
        \end{indt}
        
        \vspace{6pt}
        
        \begin{indt}{\subsection{Morphismes d'anneaux}}
            Soient $(A, +, \times)$ et $(B, +', \times')$ deux anneaux.
            
            \vspace{6pt}
            
            $\bullet$ Un morphisme d'anneaux de $(A, +, \times)$ vers $(B, +', \times')$ est un morphisme de groupes de $(A, +)$ vers $(B, +')$ tel que :
                \[ \forall (x, y) \in A,\ \eqsys{f(xy) = f(x)f(y)}{f(1) = 1} \]
            
            $\bullet$ Un isomorphisme d'anneaux est un morphisme d'anneaux bijectifs.
        \end{indt}
    \end{indt}
    
    
\end{document}
%--------------------------------------------End
