\documentclass[a4paper, 12pt, twoside]{article}


%------------------------------------------------------------------------
%
% Author                :   Lasercata
% Last modification     :   2022.03.23
%
%------------------------------------------------------------------------


%------ini
\usepackage[utf8]{inputenc}
\usepackage[T1]{fontenc}
%\usepackage[french]{babel}
%\usepackage[english]{babel}


%------geometry
\usepackage[textheight=700pt, textwidth=500pt]{geometry}


%------color
\usepackage{xcolor}
\definecolor{ff4500}{HTML}{ff4500}
\definecolor{00f}{HTML}{0000ff}
\definecolor{0ff}{HTML}{00ffff}
\definecolor{656565}{HTML}{656565}

\renewcommand{\emph}{\textcolor{ff4500}}
\renewcommand{\em}{\color{ff4500}}

\newcommand{\strong}[1]{\textcolor{ff4500}{\bf #1}}
\newcommand{\st}{\color{ff4500}\bf}


%------Code highlighting
\usepackage{listings}

\definecolor{cbg}{HTML}{272822}
\definecolor{cfg}{HTML}{ececec}
\definecolor{ccomment}{HTML}{686c58}
\definecolor{ckw}{HTML}{f92672}
\definecolor{cstring}{HTML}{e6db72}
\definecolor{cstringlight}{HTML}{98980f}
\definecolor{lightwhite}{HTML}{fafafa}

\lstdefinestyle{DarkCodeStyle}{
    backgroundcolor=\color{cbg},
    commentstyle=\itshape\color{ccomment},
    keywordstyle=\color{ckw},
    numberstyle=\tiny\color{cbg},
    stringstyle=\color{cstring},
    basicstyle=\ttfamily\footnotesize\color{cfg},
    breakatwhitespace=false,
    breaklines=true,
    captionpos=b,
    keepspaces=true,
    numbers=left,
    numbersep=5pt,
    showspaces=false,
    showstringspaces=false,
    showtabs=false,
    tabsize=4,
    xleftmargin=\leftskip
}

\lstdefinestyle{LightCodeStyle}{
    backgroundcolor=\color{lightwhite},
    commentstyle=\itshape\color{ccomment},
    keywordstyle=\color{ckw},
    numberstyle=\tiny\color{cbg},
    stringstyle=\color{cstringlight},
    basicstyle=\ttfamily\footnotesize\color{cbg},
    breakatwhitespace=false,
    breaklines=true,
    captionpos=b,
    keepspaces=true,
    numbers=left,
    numbersep=10pt,
    showspaces=false,
    showstringspaces=false,
    showtabs=false,
    tabsize=4,
    frame=L,
    xleftmargin=\leftskip
}

%\lstset{style=DarkCodeStyle}
\lstset{style=LightCodeStyle}

%Usage : \begin{lstlisting}[language=Caml] ... \end{lstlisting}


%-------make the table of content clickable
\usepackage{hyperref}
\hypersetup{
    colorlinks,
    citecolor=black,
    filecolor=black,
    linkcolor=black,
    urlcolor=black
}
%Uncomment this and comment above for dark mode
% \hypersetup{
%     colorlinks,
%     citecolor=white,
%     filecolor=white,
%     linkcolor=white,
%     urlcolor=white
% }


%------pictures
\usepackage{graphicx}
%\usepackage{wrapfig}

\usepackage{tikz}
\usetikzlibrary{shapes.geometric}


%------tabular
%\usepackage{color}
%\usepackage{colortbl}
%\usepackage{multirow}


%------Physics
%---Packages
%\usepackage[version=4]{mhchem} %$\ce{NO4^2-}$

%---Commands
\newcommand{\link}[2]{\mathrm{#1} \! - \! \mathrm{#2}}
\newcommand{\pt}[1]{\cdot 10^{#1}} % Power of ten
\newcommand{\dt}[2][t]{\dfrac{\mathrm d #2}{\mathcal d #1}} % Derivative


%------math
%---Packages
%\usepackage{textcomp}
%\usepackage{amsmath}
\usepackage{amssymb}
\usepackage{mathtools} % For abs
\usepackage{stmaryrd} %for \llbracket and \rrbracket
\usepackage{mathrsfs} %for \mathscr{x} (different from \mathcal{x})

%---Commands
%-Sets
\newcommand{\N}{\mathbb{N}} %set N
\newcommand{\Z}{\mathbb{Z}} %set Z
\newcommand{\Q}{\mathbb{Q}} %set Q
\newcommand{\R}{\mathbb{R}} %set R
\newcommand{\C}{\mathbb{C}} %set C
\newcommand{\U}{\mathbb{U}} %set U
\newcommand{\seg}[2]{\left[ #1\ ;\ #2 \right]}
\newcommand{\nset}[2]{\left\llbracket #1\ ;\ #2 \right\rrbracket}

%-Exponantial / complexs
\newcommand{\e}{\mathrm{e}}
\newcommand{\cj}[1]{\overline{#1}} %overline for the conjugate.

%-Vectors
\newcommand{\vect}{\overrightarrow}
\newcommand{\veco}[3]{\displaystyle \vect{#1}\binom{#2}{#3}} %vector + coord

%-Limits
\newcommand{\lm}[2][{}]{\lim\limits_{\substack{#2 \\ #1}}} %$\lm{x \to a} f$ or $\lm[x < a]{x \to a} f$
\newcommand{\Lm}[3][{}]{\lm[#1]{#2} \left( #3 \right)} %$\Lm{x \to a}{f}$ or $\Lm[x < a]{x \to a}{f}$
\newcommand{\tendsto}[1]{\xrightarrow[#1]{}}

%-Integral
\newcommand{\dint}[4][x]{\displaystyle \int_{#2}^{#3} #4 \mathrm{d} #1} %$\dint{a}{b}{f(x)}$ or $\dint[t]{a}{b}{f(t)}$

%-left right
\newcommand{\lr}[1]{\left( #1 \right)}
\newcommand{\lrb}[1]{\left[ #1 \right]}
\newcommand{\set}[1]{\left\{ #1 \right\}}
\newcommand{\abs}[1]{\left\lvert #1 \right\rvert} % abs{x} -> |x|
\newcommand{\floor}[1]{\left\lfloor #1 \right\rfloor}
\newcommand{\ceil}[1]{\left\lceil #1 \right\rceil}
\newcommand{\lrangle}[1]{\left\langle #1 \right\rangle}

%-Others
\newcommand{\para}{\ /\!/\ } %//
\newcommand{\ssi}{\ \Leftrightarrow \ }
\newcommand{\eqsys}[2]{\begin{cases} #1 \\ #2 \end{cases}}

\newcommand{\med}[2]{\mathrm{med} \left[ #1\ ;\ #2 \right]}  %$\med{A}{B} -> med[A ; B]$
\newcommand{\Circ}[2]{\mathscr{C}_{#1, #2}}

\renewcommand{\le}{\leqslant}
\renewcommand{\ge}{\geqslant}


%------commands
%---to quote french text
\newcommand{\simplecit}[1]{\guillemotleft$\;$#1$\;$\guillemotright}
\newcommand{\cit}[1]{\simplecit{\textcolor{656565}{#1}}}
\newcommand{\quo}[1]{\cit{\it #1}}

%---to indent
\newcommand{\ind}[1][20pt]{\advance\leftskip + #1}
\newcommand{\deind}[1][20pt]{\advance\leftskip - #1}

%---to indent a text
\newcommand{\indented}[2][20pt]{\par \ind[#1] #2 \par \deind[#1]}
\newenvironment{indt}[2][20pt]{#2 \par \ind[#1]}{\par \deind} %Titled indented env

%---title
\newcommand{\thetitle}[2]{\begin{center}\textbf{{\LARGE \underline{\emph{#1} :}} {\Large #2}}\end{center}}

%---parts
%-I
\newcommand{\mainpart}[2][$\!\!$]{\underline{\large \textbf{\emph{\textit{#1} #2}}}}
\newcommand{\bmainpart}[2][$\!\!$]{\underline{\large \textbf{\textit{#1} #2}}}
%-A
\newcommand{\subpart}[2][$\!\!$]{\underline{\bf \textit{#1} #2}}
%-1
\newcommand{\subsubpart}[2][$\!\!$]{\underline{\textsl{#1} #2}}
%-a
\newcommand{\subsubsubpart}[2][$\!\!$]{\underline{\it #1 #2}}

\setcounter{section}{-1} %Start numbering from 0


%------page style
\usepackage{fancyhdr}
\usepackage{lastpage}

\setlength{\headheight}{18pt}
\setlength{\footskip}{50pt}

\pagestyle{fancy}
\fancyhf{}
\fancyhead[LE, RO]{\textit{\today}}
\fancyhead[RE, LO]{\large{\textsl{\emph{\texttt{\jobname}}}}}

\fancyfoot[RO, LE]{\textit{\texttt{\textcolor{black}{Page \thepage /}\pageref{LastPage}}}} %Change 'black' to 'white' for dark mode
\fancyfoot[LO, RE]{\includegraphics[scale=0.12]{/home/lasercata/Pictures/1.images_profil/logo/mieux/lasercata_logo_fly_fond_blanc.png}}

% For dark mode :
%/home/lasercata/Pictures/1.images_profil/logo/mieux/lasercata_logo_fly.png


%------init lengths
\setlength{\parindent}{0pt} %To avoid using \noindent everywhere.
\setlength{\parskip}{3pt}


%---------------------------------Begin Document
\begin{document}
    
    %For dark mode :
    % \pagecolor{black}
    % \color{white}
    
    \thetitle{Maths}{Espaces vectoriels}
    
    \tableofcontents
    \newpage
    
    
    Dans tout ce qui suit, $K$ désigne un corps (par exemple $\R$ ou $\C$), $I$, $J$ des ensembles d'indexation (par exemple $\N$), et $E$ désigne un $K$-espace vectoriel.
    
    
    
    \begin{indt}{\section{Familles}}
        
        \begin{indt}{\subsection{Définition (\textit{famille})}}
            Soit $X$ un ensemble.
            
            Une famille d'éléments de $X$ indexée par $I$ est une application $x : I \longrightarrow X$, notée $(x_k)_{k \in I}$, où $\forall k \in I,\ x_k = x(k)$.
            
            \vspace{6pt}
            
            Cas particulier : si $I = \nset 1 n$, avec $n \in \N$, la famille $(x_k)_{k \in \nset 1 n} \subset X$ est une suite, et est souvent confondue avec le $n$-uplet $(x_1, \ldots, x_n) \in X^n$.
        \end{indt}
        
        \vspace{12pt}
        
        \begin{indt}{\subsection{Définition (\textit{famille à support fini})}}
            Une famille $(x_k)_{k \in I} \subset X$ est dite à \textit{support fini} si
                \[ \mathrm{card}\set{x_k \in X\ \left| \vphantom{\dfrac a a} \right.\ k \in I,\ x_k \neq 0} < +\infty \]
            \textit{i.e} si au plus un nombre fini de $x_k$ sont non nuls.
        \end{indt}
        
        \vspace{12pt}
        
        \begin{indt}{\subsection{Définition (\textit{sous-famille, sur-famille})}}
            Soit $\mathcal F = (x_k)_{k \in I} \subset X$ une famille.
            
            \vspace{6pt}
            
            Une \textit{sous-famille} de $\mathcal F$ est une famille $(x_k)_{k \in J} \subset X$, avec $J \subset I$.
            
            Une \textit{sur-famille} de $\mathcal F$ est une famille $(x_k)_{k \in J} \subset X$, avec $I \subset J$.
        \end{indt}
        
    \end{indt}
    
    \vspace{12pt}
    
    \begin{indt}{\section{Définitions}}
        
        \begin{indt}{\subsection{Définition (\textit{Espace vectoriel})}}
            Soit $K$ un corps, $(E,\ +)$ un groupe commutatif, et $\cdot : K \times E \longrightarrow E$ une application.
            
            \vspace{6pt}
            
            \begin{indt}{Un $K$-espace vectoriel est un triplet $(E,\ +,\ \cdot)$, avec $\forall (x, y) \in E^2,\ \forall (\lambda, \mu) \in K^2$ :}
                (1) $\lambda \cdot (\mu \cdot x) = (\lambda \mu) \cdot x$
                
                (2) $(\lambda + \mu) \cdot x = \lambda \cdot x + \mu \cdot x$
                
                (3) $\lambda \cdot (x + y) = \lambda \cdot x + \lambda \cdot y$
                
                (4) $1 \cdot x = x$
            \end{indt}
            
            \vspace{12pt}
            
            Les éléments de $K$ sont appelés les \textit{scalaires}, et ceux de $E$ les \textit{vecteurs}.
        \end{indt}
        
        \vspace{12pt}
        
        \begin{indt}{\subsection{Propriétés}}
            Soit $(E, +, \cdot)$ un $K$-espace vectoriel.
            
            Alors
            $
                \forall (\lambda, x) \in K \times E,\
                \left|
                \begin{array}{l}
                    0 \cdot x = 0
                    \\
                    \lambda \cdot 0 = 0
                \end{array}
                \right.
            $
            
            $\forall x \in E,\ (-1) \cdot x = -x$
        \end{indt}
        
        \vspace{12pt}
        
        \begin{indt}{\subsection{Définition (\textit{sous-espace vectoriel})}}
            Soit $(E, +, \cdot)$ un $K$-espace vectoriel.
            
            \begin{indt}{L'ensemble $F$ est un \textit{sous-espace vectoriel} de $E$ si :}
                (1) $F \in \mathcal P(E) \setminus \varnothing$
                
                (2) $\forall (\lambda, x) \in K \times F,\ \lambda x \in F$
                
                (3) $\forall (x, y) \in F^2,\ x + y \in F$
            \end{indt}
        \end{indt}
            
        \vspace{12pt}
        
        \begin{indt}{\subsection{Propriétés}}
            Soit $(E, +, \cdot)$ un $K$-espace vectoriel.
            
            \vspace{6pt}
            
            $\bullet$ Si $F$ est un sous-espace vectoriel de $E$, alors $0 \in F$, et $F$ est un $K$-espace vectoriel pour les lois induites.
            
            \vspace{6pt}
            
            $\bullet$ Soit $F \in \mathcal P(E) \setminus \varnothing$.
            
            Alors $F$ est un sous-espace vectoriel de $E$ $\ssi$ $F$ est un sous-groupe de $(E, +)$ tel que $\forall (\lambda, x) \in K \times F,\ \lambda \cdot x \in F$ (stable par multiplication par les scalaires).
            
            \vspace{12pt}
            
            $\bullet$ Soit $(F_k)_{k \in I}$ une famille de sous-espaces vectoriels de $E$. Alors
                \[ \bigcap_{k \in I} F_k \]
            est un sous-espace vectoriel de $E$.
        \end{indt}
        
        \vspace{12pt}
        
        \begin{indt}{\subsection{Caractérisation d'un sous-espace vectoriel}}
            Soit $(E, +, \cdot)$ un $K$-espace vectoriel, et $F \in \mathcal P(E) \setminus \varnothing$.
            
            \vspace{6pt}
            
            Alors $F$ est un sous-espace vectoriel de $E$ $\ssi$
                \[ \forall ((\lambda, \mu), (x, y)) \in K^2 \times F^2,\ \lambda x + \mu y \in F \]
        \end{indt}
        
    \end{indt}
    
    \vspace{12pt}
    
    \begin{indt}{\section{Combinaisons linéaires}}
        
        \begin{indt}{\subsection{Définition (combinaisons linéaires)}}
            Soit $(E, +, \cdot)$ un $K$-espace vectoriel, et $\mathcal F = (x_k)_{k \in I} \subset E$ une famille de $E$.
            
            Les combinaisons linéaires de la famille $\mathcal F$ sont les vecteurs de la forme :
                \[ \sum_{k \in I} \lambda_k x_k \]
            où $(\lambda_k)_{k \in I} \subset K$ est une famille de scalaires à \textit{support fini}.
            
            \vspace{6pt}
            
            On note l'ensemble des combinaisons linéaires :
                \[ \mathrm{vect}(x_k)_{k \in I} = \set{\sum_{k \in I} \lambda_k x_k\ |\ (\lambda_k)_{k \in I} \subset K} \]
        \end{indt}
        
        \vspace{12pt}
        
        \begin{indt}{\subsection{Définition (sous-espace vectoriel engendré par une partie)}}
            Soit $E$ un $K$-espace vectoriel, et $X \subset E$.
            
            Alors le sous-espace vectoriel engendré par $X$ est l'ensemble des combinaisons linéaires des vecteurs de $X$, noté $\mathrm{vect}(X)$.
        \end{indt}
        
        \vspace{12pt}
        
        \begin{indt}{\subsection{Propriétés}}
            \label{2.3}
            
            Soit $E$ un $K$-espace vectoriel, et $x \subset E$
            
            $\bullet$ Soit $F$ un sous-espace vectoriel de $E$.
            
            Alors $F = \mathrm{vect}(X) \ssi \eqsys{X \subset F}{\forall G \supset X\ \text{sous-espace vectoriel de}\ E,\ F \subset G}$
            
            \vspace{12pt}
            
            $\bullet$ $\forall (x_k)_{k \in I}, (y_k)_{k \in J} \subset E,\ (x_k)_{k \in I} \subset \mathrm{vect}(y_k)_{k \in J} \Rightarrow \mathrm{vect}(x_k)_{k \in I} \subset \mathrm{vect}(y_k)_{k \in J}$
            
            \vspace{12pt}
            
            $\bullet$ Soit $(x_k)_{k \in I} \subset E$, et $i \in I$.
            
            Alors $\forall (\lambda_k)_{k \in I} \subset K\ |\ \forall k \in I,\ \lambda_k \neq 0$ :
                \[ \mathrm{vect}(x_k)_{k \in I} = \mathrm{vect}\lr{x_i + \sum_{j \in I \setminus \set i} \lambda_j x_j,\ x_k}_{k \in I \setminus \set i} \]
            \textit{i.e} $\mathrm{vect}(x_k)_{k \in I}$ ne change pas si à l’un des vecteurs on rajoute une combinaison linéaire des autres, et :
                \[ \mathrm{vect}(x_k)_{k \in I} = \mathrm{vect}(\lambda_k x_k)_{k \in I} \]
        \end{indt}
        
    \end{indt}
    
    \vspace{12pt}
    
    \begin{indt}{\section{Familles libres, génératrices, bases}}
        
        \begin{indt}{\subsection{Familles génératrices}}
            \begin{indt}{\subsubsection{Définition (\textit{famille génératrice})}}
                Soit $E$ un $K$-espace vectoriel.
                
                \vspace{6pt}
                
                Un sous-espace vectoriel $A$ de $E$ est engendré par une famille $(x_k)_{k \in I} \subset A$ si
                    \[ A = \mathrm{vect}(x_k)_{k \in I} \]
                ou encore si
                    \[ \forall x \in A,\ \exists (\lambda_k)_{k \in I} \subset K\ |\ x = \sum_{k \in I} \lambda_k x_k \]
                
                La famille $(x_k)_{k \in I}$ est alors une \textit{famille génératrice} de $A$.
            \end{indt}
            
            \vspace{12pt}
            
            \begin{indt}{\subsubsection{Propriétés}}
                Soit $E$ un $K$-espace vectoriel.
                
                \vspace{6pt}
                
                $\bullet$ Une sur-famille d’une famille génératrice de $E$ est une famille génératrice de $E$.
                
                \vspace{12pt}
                
                $\bullet$ Soit $(x_k)_{k \in I} \subset E$ une famille génératrice de $E$. Alors une famille $(y_k)_{k \in J} \subset E$ est génératrice de $E$ si, et seulement si :
                    \[ \forall k \in I, x_k \in \mathrm{vect}(y_k)_{k \in J} \]
                Autrement dit, si $\mathcal F$ est une famille génératrice de $E$, alors une famille $\mathcal G$ est génératrice de $E$ si, et seulement si :
                    \[ \mathcal F \subset \mathrm{vect}(\mathcal G) \]
                
                \vspace{12pt}
                
                $\bullet$ Soit $\mathcal F = (x_k)_{k \in I} \subset E$ une famille génératrice de $E$, et $(x_k)_{k \in J} \subset E$ une sous-famille de $\mathcal F$, telle que
                    \[ \forall k \in J,\ x_k \in \mathrm{vect}(x_k)_{k \in I \setminus J} \]
                Alors $(x_k)_{k \in I \setminus J}$ est une famille génératrice de $E$
            \end{indt}
        \end{indt}
        
        \vspace{12pt}
        
        \begin{indt}{\subsection{Familles libres, liées}}
            \begin{indt}{\subsubsection{Définition (\textit{famille libre, liée})}}
                Une famille finie $(x_k)_{k \in I} \subset E$ est \textit{libre}, ou \textit{indépendante linéairement} si
                    \[ \forall (\lambda_k)_{k \in I} \subset K,\ \sum_{k \in I} \lambda_k x_k = 0\ \Rightarrow\ \forall k \in I,\ \lambda_k = 0 \]
                
                Dans le cas contraire, \textit{i.e} si
                    \[
                        \exists (\lambda_k)_{k \in I} \subset K
                        \left|
                        \begin{array}{l}
                            \exists i \in I\ |\ \lambda_i \neq 0
                            \vspace{6pt}
                            \\
                            \displaystyle \sum_{k \in I} \lambda_k x_k = 0
                        \end{array}
                        \right.
                    \]
                alors elle est \textit{liée}.
                
                \vspace{12pt}
                
                Une famille infinie de $E$ est libre si toute ses sous-familles finies sont libres.
                
                Elle est liée si elle admet une sous-famille finie liée.
            \end{indt}
            
            \vspace{12pt}
            
            \begin{indt}{\subsubsection{Propriétés}}
                $\bullet$ Une famille $(P_k)_{k \in I} \subset K[X]\ |\ \forall (i, j) \in I^2,\ i \neq j \Rightarrow \deg P_i \neq \deg P_j$ est libre.
                
                \vspace{12pt}
                
                $\bullet$ Une famille $(x_k)_{k \in I} \subset E$ est liée $\ssi \exists i \in I\ |\ x_i = \mathrm{vect}(x_k)_{k \in I \setminus J}$
                
                \vspace{12pt}
                
                $\bullet$ Toute sous-famille d'une famille libre est libre, toute sur-famille d'une famille liée est liée.
                
                \vspace{12pt}
                
                $\bullet$ Soit une famille $\mathcal F = (x_k)_{k \in I} \subset E$. Alors $\mathcal F$ est libre si, et seulement si
                    \[ \forall (\lambda_k)_{k \in I}, (\mu_k)_{k \in I} \subset K,\ \sum_{k \in I} \lambda_k x_k = \sum_{k \in I} \mu_k x_k \Rightarrow \forall k \in I,\ \lambda_k = \mu_k \]
                (les familles de scalaires doivent être à support fini) ou encore si, et seulement si
                    \[ \forall x \in \mathrm{vect}(x_k)_{k \in I},\ \exists! (\lambda_k)_{k \in I} \subset K\ |\ x = \sum_{k \in I} \lambda_k x_k \]
            \end{indt}
        \end{indt}
        
        \vspace{12pt}
        
        \begin{indt}{\subsection{Bases}}
            \begin{indt}{\subsubsection{Définition (\textit{Base})}}
                Soit $E$ un $K$-espace vectoriel.
                
                \vspace{6pt}
                
                Une \textit{base} de $E$ est une famille libre et génératrice de $E$.
            \end{indt}
            
            \vspace{12pt}
            
            \begin{indt}{\subsubsection{Propriétés}}
                $\bullet$ Soit $\mathcal F = (e_k)_{k \in I} \subset E$.
                Alors $\mathcal F$ est une base de $E$ si, et seulement si
                    \[
                        \forall x \in E,\
                        \exists! (x_k)_{k \in I} \subset K\ |\
                        x = \sum_{k \in I} x_k e_k
                    \]
                où la famille de scalaires est à support fini.
                Dans ce cas, si $I = \nset 1 p$, $(x_1, \ldots, x_p)$ sont les \textit{composantes} de $x$ dans la base $\mathcal F$, souvent noté verticalement :
                $
                    \begin{pmatrix}
                        x_1
                        \\
                        \vdots
                        \\
                        x_p
                    \end{pmatrix}
                $
                
                \vspace{12pt}
                
                $\bullet$ Soit $(v_k)_{k \in \nset 1 p}$ une base de $E$, et $x \in E$ tel que
                    \[ \exists (\lambda_k)_{k \in \nset 2 p} \subset K\ |\ x = v_1 + \sum_{k = 2}^p \lambda_k v_k \]
                Alors $(x, v_2, \ldots, v_p)$ est une base de $E$.
                
                \vspace{12pt}
                
                $\bullet$ Soient $(e_k)_{k \in I} \subset E,\ (f_k)_{k \in J} \subset F$ des bases respectives de $E$ et $F$.
                
                Alors la famille
                    \[ \lr{ \vphantom{\frac a a} (e_k, 0_F)_{k \in I}, (f_k, 0_E)_{k \in J}} \]
                est une base de $E \times F$.
            \end{indt}
        \end{indt}
        
    \end{indt}
    
    \vspace{12pt}
    
    \begin{indt}{\section{Somme de sous-espaces vectoriels}}
        
        \begin{indt}{\subsection{Définition (somme finie de sous-espaces vectoriels)}}
            Soient $E$ un $K$-espace vectoriel, $n \in \N^*$, et $(E_k)_{k \in \nset 1 n}$ des sous-espaces vectoriels de $E$.
            
            La somme de ces sous-espaces vectoriels est le sous espace vectoriel :
                \[ \sum_{k = 1}^n E_k = \set{ \sum_{k = 1}^n x_k\ |\ \forall k \in \nset 1 n, x_k \in E_k } \]
            
            Par exemple pour $n = 2$ :
                \[ E_1 + E_2 = \set{x + y\ |\ (x, y) \in E_1 \times E_2 \vphantom{\tfrac a a}} \]
        \end{indt}
        
        \vspace{12pt}
        
        \begin{indt}{\subsection{Propriétés}}
            $\bullet$ Soit $n \in \N^*$, et $(E_k)_{k \in \nset 1 n}$ des sous-espaces vectoriels de $E$. Alors
                \[ \mathrm{vect}\lr{\bigcup_{k = 1}^n E_k} = \sum_{k = 1}^n E_k \]
        \end{indt}
        
        \vspace{12pt}
        
        \begin{indt}{\subsection{Définition (\textit{somme directe})}}
            $\bullet$ Soient $A, B$ deux sous-espaces vectoriels de $E$.
            
            $A$ et $B$ sont en \textit{somme directe}, noté $A + B = A \oplus B$, si :
                \[ A + B = A \oplus B \ssi \forall x \in A + B,\ \exists! (a, b) \in A \times B\ |\ x = a + b \]
            
            
            $\bullet$ Soient $n \in \N^*$, et $(E_k)_{k \in \nset 1 n}$ des sous-espaces vectoriels de $E$.
            
            Si la somme de ces sous-espaces vectoriels est \textit{directe}, on note
                \[ \sum_{k = 1}^n E_k = \bigoplus_{k = 1}^n E_k \]
            et on a:
                \[
                    \sum_{k = 1}^n E_k = \bigoplus_{k = 1}^n E_k
                    \ssi
                    \forall x \in \sum_{k = 0}^n E_k,\
                    \exists! (x_k)_{k \in \nset 1 n}\
                    \left|
                    \begin{array}{l}
                        \forall k \in \nset 1 n,\ x_k \in E_k
                        \vspace{6pt}
                        \\
                        \displaystyle x = \sum_{k = 1}^n x_k
                    \end{array}
                    \right.
                \]
            \textit{i.e} si tout vecteur de la somme se décompose de façon unique comme somme de vecteurs de chaque sous-espace vectoriel.
        \end{indt}
        
        \vspace{12pt}
        
        \begin{indt}{\subsection{Caractérisation d'une somme directe}}
            \begin{indt}{\subsubsection{Cas de deux vecteurs}}
                Soient $A, B$ deux sous-espaces vectoriel de $E$. Alors :
                    \[ A + B = A \oplus B \ssi A \cap B = \set 0 \]
            \end{indt}
            
            \vspace{12pt}
            
            \begin{indt}{\subsubsection{Cas d'un nombre fini de vecteurs}}
                Soit $n \in \N^*$, et $(E_k)_{k \in \nset 1 n}$ des sous-espaces vectoriels de $E$. Alors :
                    \[
                        \begin{array}{c}
                            \displaystyle \sum_{k = 1}^n E_k = \bigoplus_{k = 1}^n E_k
                            \vspace{6pt}
                            %\\
                            \displaystyle
                            \ssi
                            \forall (x_k)_{k \in \nset 1 n}\ |\
                            \forall k \in \nset 1 n,\ x_k \in E_k,\
                            \\
                            \displaystyle
                            \sum_{k = 0}^n x_k = 0\ \Rightarrow\ \forall k \in \nset 1 n,\ x_k = 0
                        \end{array}
                    \]
                
                Remarque : la condition
                    $\displaystyle \bigcup_{k \in \nset 1 n} E_k = \set 0 $
                ne suffit pas (exemple : trois droites dans le plan).
            \end{indt}
        \end{indt}
        
        \vspace{12pt}
        
        \begin{indt}{\subsection{Définition (sous-espaces \textit{supplémentaires})}}
            Soient $A, B$ deux sous-espaces vectoriels de $E$.
            
            \vspace{6pt}
            
            1. Les sous-espaces vectoriels $A$ et $B$ sont \textit{supplémentaires} si
                \[ E = A \oplus B \]
            \textit{i.e} si
                \[ \eqsys{E = A + B}{A + B = A \oplus B} \]
            
            \vspace{12pt}
            
            2. Un \textit{supplémentaire} de $A$ dans $E$ est un sous-espace vectoriel $G$ de $E$ tel que $E = A \oplus G$.
        \end{indt}
        
        \vspace{12pt}
        
        \begin{indt}{\subsection{Propriétés}}
            $\bullet$ Soient $n \in \N^*$, $(E_k)_{k \in \nset 1 n}$ des sous-espaces vectoriels de $E$, et $(\mathcal F_k)_{k \in \nset 1 n}$ des familles génératrices respectives de $E_1, \ldots, E_n$.
            
            Alors : %$\mathcal F \cup \mathcal G$ est une famille génératrice de $A + B$, \textit{i.e}
                \[ \sum_{k = 1}^n E_k = \mathrm{vect}\lr{\bigcup_{k \in \nset 1 n} \mathcal F_k} \]
            \textit{i.e} $\displaystyle \bigcup_{k \in \nset 1 n} \mathcal F_k$ est une famille génératrice de $\displaystyle \sum_{k = 1}^n E_k$.
            
            \vspace{12pt}
            
            De plus, si $\displaystyle \sum_{k = 1}^n E_k = \bigoplus_{k = 1}^n E_k$, alors la famille $\displaystyle \bigcup_{k \in \nset 1 n} \mathcal F_k$ est libre (et génératrice de la somme).
            
            %Et si $A + B = A \oplus B$, alors la famille $\mathcal F \cup \mathcal G$ est libre (et génératrice de $A \oplus B$).
            
            \vspace{12pt}
            
            $\bullet$ Soient $n \in \N^*$, $(E_k)_{k \in \nset 1 n}$ des sous-espaces vectoriels de $E$ tels que $\displaystyle \sum_{k = 1}^n E_k = \bigoplus_{k = 1}^n E_k$, et $(\mathcal F_k)_{k \in \nset 1 n}$ des bases respectives de $E_1, \ldots, E_n$.
            
            \vspace{6pt}
            
            Alors $\displaystyle \bigcup_{k \in \nset 1 n} \mathcal F_k$ est une base de $\displaystyle \bigoplus_{k = 1}^n E_k$.
        \end{indt}
        
    \end{indt}
    
    \vspace{12pt}
    
    \begin{indt}{\section{Espaces vectoriels de dimension finie}}
        
        \begin{indt}{\subsection{Définition (espace vectoriel de dimension finie)}}
            Un espace vectoriel est de \textit{dimension finie} si il admet une famille génératrice finie. Sinon, il est de \textit{dimension infinie}.
        \end{indt}
        
        \vspace{12pt}
        
        \begin{indt}{\subsection{Lemme}}
            Soient $\mathcal F$ une famille finie libre de $E$, et $\mathcal G$ une famille finie génératrice de $E$.
            
            Alors
                \[ \mathrm{card}( \mathcal F) \le \mathrm{card}(\mathcal G) \]
        \end{indt}
        
        \vspace{12pt}
        
        \begin{indt}{\subsection{Théorème}}
            Soit $E$ un espace vectoriel de dimension finie.
            
            \begin{indt}{Alors :}
                (1) $E$ admet une base finie ;
                
                (2) Toutes les bases de $E$ sont finies, et ont le même nombre d'éléments.
            \end{indt}
        \end{indt}
        
        \vspace{12pt}
        
        \begin{indt}{\subsection{Définition (\textit{dimension d'un espace vectoriel})}}
            Soit $E$ un $K$-espace vectoriel de dimension finie.
            
            La \textit{dimension} de $E$, notée $\dim(E)$, est le nombre d'éléments de ses bases.
        \end{indt}
        
        \vspace{12pt}
        
        \begin{indt}{\subsection{Propriétés}}
            Soient $E$ un $K$-espace vectoriel de dimension finie, et $\mathcal F$ une famille de $E$.
            
            Alors, si $\mathcal F$ est libre :
                \[ \mathrm{card}(\mathcal F) \le \dim(E) \]
            et $\mathcal F$ est une base si, et seulement si $\mathrm{card}(\mathcal F) = \dim(E)$.
            
            Et si $\mathcal F$ est génératrice, alors
                \[ \mathrm{card}(\mathcal F) \ge \dim(E) \]
            et $\mathcal F$ est une base si, et seulement si $\mathrm{card}(\mathcal F) = \dim(E)$.
        \end{indt}
        
        \begin{indt}{\subsection{Théorème de la base extraite}}
            Soit $E$ un espace vectoriel de dimension finie, et $\mathcal F$ une famille génératrice de $E$.
            
            Alors $\exists \mathcal G \subset \mathcal F\ |\ \mathcal G$ soit une base de $E$.
        \end{indt}
        
        \vspace{12pt}
        
        \begin{indt}{\subsection{Théorème de la base incomplète}}
            Soit $E$ un espace vectoriel de dimension finie, $r \in \N^*$, $n = \dim(E)$, et $\mathcal F = (y_k)_{k \in \nset 1 r}$ une famille libre de $E$.
            
            Alors $\exists (y_k)_{k \in \nset{r + 1}{n}} \subset E \ |\ (y_k)_{k \in \nset 1 n}$ soit une base de $E$.
        \end{indt}
        
        \vspace{12pt}
        
        \begin{indt}{\subsection{Corrolaire}}
            Soit $E$ un espace vectoriel (pas nécessairement de dimension finie), $n, p \in \N^*\ |\ n < p$, et $(x_k)_{k \in \nset 1 n} \subset E$.
            
            Alors
                \[ \forall (y_k)_{k \in \nset 1 p} \subset \mathrm{vect}(x_k)_{k \in \nset 1 n},\ (y_k)_{k \in \nset 1 p}\ \text{est liée} \]
        \end{indt}
        
        \vspace{12pt}
        
        \begin{indt}{\subsection{Propriété}}
            Soit $E$ un espace vectoriel. Alors
                \[ \dim(E) = +\infty \ssi \forall n \in \N^*,\ \exists (x_k)_{k \in \nset 1 n} \subset E\ |\ (x_k)_{k \in \nset 1 n}\ \text{libre} \]
        \end{indt}
        
    \end{indt}
    
    \vspace{12pt}
    
    \begin{indt}{\section{Dimension d'un sous-espace vectoriel en dimension finie}}
        
        \begin{indt}{\subsection{Propriété}}
            Soit $E$ un espace vectoriel de dimension finie, et $F$ un sous-espace vectoriel de $E$.
            
            Alors
                \[ \dim(F) \le \dim(E) \]
            et
                \[ E = F \ssi \dim(F) = \dim(E) \]
        \end{indt}
        
        \vspace{12pt}
        
        \begin{indt}{\subsection{Définition (\textit{hyperplan})}}
            Un \textit{hyperplan} d'un espace vectoriel $E$ de dimension finie est un sous-espace vectoriel $F$ de $E$ tel que
                \[ \dim(F) = \dim(E) - 1 \]
        \end{indt}
        
        \begin{indt}{\subsection{Propriétés}}
            Soit $E$ un espace vectoriel de dimension finie, et $F, G$ deux sous-espaces vectoriels de $E$.
            
            Alors
                \[
                    \begin{array}{rcl}
                        F = G &\ssi& \eqsys{\dim(F) = \dim(G)}{F \subset G}
                        \vspace{6pt}
                        \\
                        &\ssi& \eqsys{\dim(F) = \dim(G)}{\exists (x_k)_{k \in I} \subset G\ |\ F = \mathrm{vect}(x_k)_{k \in I}}
                    \end{array}
                \]
            
%             De plus,
%                 \[
%                     F \subset G
%                     \ssi
%                     \exists (x_k)_{k \in I} \subset G\ |\
%                     F = \mathrm{vect}(x_k)_{k \in I}
% %                     \left|
% %                     \begin{array}{l}
% %                         F = \mathrm{vect}(x_k)_{k \in I}
% %                         \\
% %                         (x_k)_{k \in I} \subset G
% %                     \end{array}
% %                     \right.
%                 \]
        \end{indt}
        
        \vspace{12pt}
        
        \begin{indt}{\subsection{Définition (système d'équation dans une base)}}
            Soit $E$ un espace vectoriel de dimension finie, $n = \dim(E)$, $\mathcal B$ une base de $E$, et $F$ un sous-espace vectoriel de $E$.
            
            Un \textit{système d'équation} de $F$ dans la base $\mathcal B$ est un système $S$ de $p \in \N$ équations ($p \le n$) à $n$ inconnues $(x_k)_{k \in \nset 1 n}$ tel que $\forall v \in E,\ v \in F \ssi$ ses composantes dans la base $\mathcal B$ sont solutions de $S$.
        \end{indt}
        
    \end{indt}
    
    \vspace{12pt}
    
    \begin{indt}{\section{Rang d'une famille de vecteurs}}
        
        \begin{indt}{\subsection{Définition (\textit{rang})}}
            Soit $E$ un $K$-espace vectoriel, et $\mathcal F = (x_k)_{k \in I} \subset E$.
            
            Le \textit{rang} de la famille de vecteurs $\mathcal F$, noté $\mathrm{rang}(x_k)_{k \in I}$, est la dimension du sous-espace vectoriel engendré par les vecteurs de $\mathcal F$, \textit{i.e} :
                \[ \mathrm{rang}(x_k)_{k \in I} = \dim\!\lr{\mathrm{vect}(x_k)_{k \in I} \vphantom{\tfrac a a}} \]
        \end{indt}
        
        \vspace{12pt}
        
        \begin{indt}{\subsection{Propriétés}}
            $\bullet$ Soit $E$ un $K$-espace vectoriel, et $(x_k)_{k \in I} \subset E$. Alors :
                \[ \mathrm{rang}(x_k)_{k \in I} = 0 \ssi \forall k \in I,\ x_k = 0 \]
                \[
                    \mathrm{rang}(x_k)_{k \in I} = 1
                    \ssi
                    \exists i \in I\
                    \left|
                    \begin{array}{l}
                        x_i \neq 0
                        \\
                        \forall k \in I,\ x_k \propto x_i
                    \end{array}
                    \right.
                \]
            
            ($x_k \propto x_i \ssi \exists \lambda \in K^*\ |\ x_k = \lambda x_i$)
            
            \vspace{18pt}
            
            $\bullet$ Soient $(x_k)_{k \in I} \subset E$, et $i \in I$. Alors :
                \[ x_i \in \mathrm{vect}(x_k)_{k \in I \setminus \set i} \ssi \mathrm{rang}(x_k)_{k \in I} = \mathrm{rang}(x_k)_{k \in I \setminus \set i} \]
            
            \vspace{12pt}
            
            $\bullet$ Soient $(x_k)_{k \in I}, (y_k)_{k \in J} \subset E$. Alors :
                \[ \mathrm{vect}(x_k)_{k \in I} = \mathrm{vect}(y_k)_{k \in J}\ \Rightarrow\ \mathrm{rang}(x_k)_{k \in I} = \mathrm{rang}(y_k)_{k \in J} \]
            
            \vspace{12pt}
            
            $\bullet$ Soient $p \in \N^*$, et $(x_k)_{k \in \nset 1 p} \subset E$. Alors :
                \[ \mathrm{rang}(x_k)_{k \in \nset 1 p} \le p \]
            et
                \[ \mathrm{rang}(x_k)_{k \in \nset 1 p} = p \ssi (x_k)_{k \in \nset 1 p}\ \text{libre} \]
                
            De plus, si $\exists r \in \nset 1 p\ |\ (x_k)_{k \in \nset 1 r}$ libre, alors :
                \[ \mathrm{rang}(x_k)_{k \in \nset 1 p} \ge r \]
            
            \vspace{12pt}
            
            $\bullet$ Soient $(x_k)_{k \in I} \subset E$, $i \in I$, $x \in \mathrm{vect}(x_k)_{k \in I \setminus \set i}$. Alors :
                \[ \mathrm{rang}(x_k)_{k \in I} = \mathrm{rang}(x_i + x, x_k)_{k \in I \setminus \set i} \]
            et
                \[ \forall (\lambda_k)_{k \in I} \subset K,\ \mathrm{rang}(x_k)_{k \in I} = \mathrm{rang}(\lambda_k x_k)_{k \in I} \]
            
            \textit{cf} \ref{2.3}, troisième point.
        \end{indt}
        
    \end{indt}
    
    \vspace{12pt}
    
    \begin{indt}{\section{Somme et produit de sous-espaces vectoriels en dimension finie}}
        
        \begin{indt}{\subsection{Somme de deux sous-espaces vectoriels}}
            \begin{indt}{\subsubsection{Propriétés}}
                Soit $E$ un espace vectoriel de dimension finie $n$.
                
                \vspace{6pt}
                
                $\bullet$ Soient
                $
                    \left|
                    \begin{array}{l}
                        n \in \N^*, p \in \nset 1 n
                        \vspace{3pt}
                        \\
                        (e_k)_{k \in \nset 1 n} \subset E\ \text{une base de $E$}
                        \vspace{3pt}
                        \\
                        A = \mathrm{vect}(e_k)_{k \in \nset 1 p}
                        \vspace{3pt}
                        \\
                        B = \mathrm{vect}(e_k)_{k \in \nset{p + 1}{n}}
                    \end{array}
                    \right.
                $. Alors $A \oplus B = E$.
                
                \vspace{6pt}
                
                De plus, $(e_k)_{k \in \nset 1 p}$ est une base de $A$, et $(e_k)_{k \in \nset{p + 1}{n}}$ est une base de $B$.
                
                \vspace{12pt}
                
                $\bullet$ Soient
                $
                    \left|
                    \begin{array}{l}
                        A, B \subset E\ |\ E = A \oplus B
                        \vspace{3pt}
                        \\
                        p = \dim(A), q = \dim(B)
                        \vspace{3pt}
                        \\
                        (e_k)_{k \in \nset 1 p} \subset A\ \text{une base de $A$}
                        \vspace{3pt}
                        \\
                        (e_k)_{k \in \nset{p + 1}{p + q}} \subset B\ \text{une base de $B$}
                    \end{array}
                    \right.
                $
                
                \vspace{6pt}
                
                Alors $(e_k)_{k \in \nset 1 {p + q}}$ est une base de $E$.
                
                \vspace{12pt}
                
                $\bullet$ Tout sous-espace vectoriel de $E$ admet un supplémentaire.
                
                \vspace{12pt}
                
                $\bullet$ Soient $A, B$ deux sous-espaces vectoriels de $E$ tels que $A + B = A \oplus B$. Alors :
                    \[ \dim(A \oplus B) = \dim(A) + \dim(B) \]
                
                En particulier, si $E = A \oplus B$, on a
                    \[ \dim(E) = \dim(A) + \dim(B) \]
            \end{indt}
            
            \vspace{12pt}
            
            \begin{indt}{\subsubsection{Formule de Grassmann}}
                Soient $F, G$ deux sous-espaces vectoriels d'un espace vectoriel $E$ de dimension finie.
                
                Alors :
                    \[ \dim(F + G) = \dim(F) + \dim(G) - \dim(F \cap G) \]
            \end{indt}
            
            \vspace{12pt}
            
            \begin{indt}{\subsubsection{Caractérisation des sous-espaces vectoriels supplémentaires}}
                Soient $F, G$ deux sous-espaces vectoriels d'un espace vectoriel $E$ de dimension finie.
                
                Alors :
                    \[
                        \begin{array}{rcl}
                            E = F \oplus G
                            &\ssi& \eqsys{E = F + G}{\dim(E) = \dim(F) + \dim(G)}
                            \vspace{6pt}
                            \\
                            &\ssi& \eqsys{F \cap G = \set 0}{\dim(E) = \dim(F) + \dim(G)}
                        \end{array}
                    \]
            \end{indt}
            
            \vspace{12pt}
            
            \begin{indt}{\subsubsection{Cas d'un hyperplan}}
                Soit $H$ un hyperplan de $E$, et $u \in E \setminus H$. Alors
                    \[ E = H \oplus \mathrm{vect}(u) \]
            \end{indt}
        \end{indt}
        
        \vspace{12pt}
        
        \begin{indt}{\subsection{Dimension des sommes et produits d'espaces vectoriels}}
            \begin{indt}{\subsubsection{Dimension d'un produit}}
                Soient $n \in \N^*$, et $(E_k)_{k \in \nset 1 n}$ des espaces vectoriels de dimension finie.
                
                Alors $\displaystyle \dim\!\lr{\prod_{k = 1}^n E_k} < +\infty$, et :
                    \[ \dim\!\lr{\prod_{k = 1}^n E_k} = \sum_{k = 1}^n \dim(E_k) \]
            \end{indt}
            
            \vspace{12pt}
            
            \begin{indt}{\subsubsection{Propriétés}}
                $\bullet$ Soient
                $
                    \left|
                    \begin{array}{l}
                        n \in \N^*, m \in \N
                        \vspace{3pt}
                        \\
                        (e_k)_{k \in \nset 1 n} \subset E\ \text{une base de $E$}
                        \vspace{3pt}
                        \\
                        (p_k)_{k \in \nset 0 m} \subset \N\
                        \left|
                        \begin{array}{l}
                            p_0 = 0, p_m = n
                            \\
                            \forall k \in \nset{0}{m - 1},\ p_k < p_{k + 1}
                        \end{array}
                        \right.
                        \vspace{3pt}
                        \\
                        \forall i \in \nset 1 m,\ A_i = \mathrm{vect}(e_k)_{k \in \nset{p_{i - 1} + 1}{p_i}}
                    \end{array}
                    \right.
                $
                
                Alors :
                    \[ E = \bigoplus_{i = 1}^m A_i = \bigoplus_{i = 1}^m \mathrm{vect}(e_k)_{k \in \nset{p_{i - 1} + 1}{p_i}} \]
                
                \vspace{12pt}
                
                $\bullet$
%                 Soient $m \in \N^*$, $(A_k)_{k \in \nset 1 m} \subset \mathcal P(E)$ des sous-espaces vectoriels de $E$ tels que
%                     \[ E = \bigoplus_{k = 1}^m A_k \]
%                 Soient $(p_k)_{k \in \nset 1 m} \subset N\ |\ \forall k \in \nset 1 m,\ p_k = \dim(A_k)$
                Soient
                $
                    \left|
                    \begin{array}{l}
                        m \in \N^*
                        \\
                        (A_k)_{k \in \nset 1 m} \subset \mathcal P(E)\ \text{des s-ev de $E$}\ |\ \displaystyle E = \bigoplus_{k = 1}^m A_k
                        \vspace{3pt}
                        \\
                        (p_k)_{k \in \nset 1 m} \subset N\ |\ \forall k \in \nset 1 m,\ p_k = \dim(A_k)
                        \vspace{3pt}
                        \\
                        \forall i \in \nset 1 m,\ (e_{k,i})_{k \in \nset{1}{p_i}} \subset A_i\ \text{une base de $A_i$}
                    \end{array}
                    \right.
                $
                
                \vspace{12pt}
                
                Alors $(e_{ij})_{\substack{i \in \nset 1 {p_j} \\ j \in \nset 1 m}}$ est une base de $E$.
            \end{indt}
            
            \vspace{12pt}
            
            \begin{indt}{\subsubsection{Dimension d'une somme directe}}
                Soient $E$ un espace vectoriel de dimension finie, $m \in \N^*$, $(A_k)_{k \in \nset 1 m} \subset \mathcal P(E)$ des sous-espaces vectoriels de $E$, tels que
                    \[ \sum_{k = 1}^m A_k = \bigoplus_{k = 1}^m A_k \]
                
                Alors :
                    \[ \dim\!\lr{\bigoplus_{k = 1}^m A_k} = \sum_{k = 1}^m \dim(A_k) \]
            \end{indt}
            
            \vspace{12pt}
            
            \begin{indt}{\subsubsection{Dimension d'une somme}}
                Soient $n \in \N^*$, $(E_k)_{k \in \nset 1 n} \subset \mathcal P(E)$ des sous-espaces vectoriels de dimension finie d'un espace vectoriel $E$.
                
                Alors $\displaystyle \dim\!\lr{\sum_{k = 1}^n E_k} < +\infty$, et
                    \[ \dim\!\lr{\sum_{k = 1}^n E_k} \le \sum_{k = 1}^n \dim(E_k) \]
                avec égalité si et seulement si la somme est directe.
            \end{indt}
        \end{indt}
        
    \end{indt}
    
    
    
\end{document}
%--------------------------------------------End
